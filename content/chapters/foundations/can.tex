\section{Controller Area Network (CAN bus)}

Il Controller Area Network (CAN) è uno standard di comunicazione seriale progettato originariamente da Bosch negli anni ’80 per collegare in modo efficiente le centraline elettroniche (ECU) all’interno dei veicoli, riducendo la complessità del cablaggio grazie a un’architettura di tipo bus condiviso. Il protocollo è stato successivamente standardizzato dall’ISO nella famiglia di norme ISO 11898, che ne descrivono il livello data link e il livello fisico secondo il modello OSI~\cite{iso:can}. A differenza di altre reti, il CAN adotta un paradigma multi-master e broadcast: ogni nodo può trasmettere sul bus e tutti i nodi ricevono ogni frame, decidendo localmente se elaborarlo o ignorarlo in base all’identificatore del messaggio (CAN ID). Questa modalità si adatta in modo naturale a scenari real-time in cui molteplici centraline devono condividere il loro stato.

Sul piano fisico, il CAN bus utilizza una coppia di fili intrecciati (CAN\_H e CAN\_L) e una codifica differenziale per aumentare la robustezza al rumore elettromagnetico. Lo stato logico dominante (bit 0) è rappresentato da una differenza di tensione di circa 2,5 V tra le due linee, mentre lo stato recessivo (bit 1) corrisponde a una condizione in cui entrambe le linee sono riportate a un livello intermedio tramite resistenze passive. L’utilizzo di segnali differenziali, in combinazione con le resistenze di terminazione da 120~$\Omega$ alle estremità del bus, consente di raggiungere distanze dell’ordine di decine o centinaia di metri con velocità fino a 1 Mbit/s (CAN 2.0) o fino a 5 Mbit/s nel caso di CAN FD (Flexible Data-rate).