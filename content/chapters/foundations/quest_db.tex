\section{QuestDB}

QuestDB è un database time-series open source progettato specificamente per la gestione di flussi di dati indicizzati temporalmente, con particolare attenzione a throughput di ingestion elevati e query analitiche a bassa latenza~\cite{questdb:architecture}. A differenza dei database relazionali tradizionali, QuestDB adotta uno storage column-oriented e un modello dati nativamente temporale: il timestamp è un tipo primitivo e può fungere sia da chiave di partizionamento sia da dimensione centrale per le operazioni di query. Questo design si presta in modo naturale all’analisi di serie storiche, come dati di telemetria, metriche di sistema, segnali finanziari o misure IoT.

L’architettura interna di QuestDB si basa su una pipeline a tre livelli: un livello di ingest caldo tramite Write-Ahead Log (WAL), uno storage binario nativo columnar e, opzionalmente, uno strato di persistenza in formato Parquet su storage locale o remoto. I dati vengono inizialmente scritti in log transazionali, quindi riordinati e deduplicati in base al timestamp e infine materializzati nel formato colonnare partizionato nel tempo. Questo approccio consente di combinare durabilità, capacità di ingestion ad alta velocità e interoperabilità con ecosistemi esterni (ad esempio data lake basati su formato Parquet).

QuestDB implementa un’interfaccia SQL estesa con primitive specifiche per le serie temporali, come ASOF JOIN, SAMPLE BY, LATEST ON e funzioni di downsampling e aggregazione temporale. Queste estensioni semplificano la formulazione di query tipiche del dominio time-series (calcolo di medie mobili, aggregazioni su finestre temporali, ricostruzione dello stato più recente di un sistema) mantenendo una sintassi familiare a chi è abituato al paradigma relazionale.

Dal punto di vista dell’integrazione, QuestDB supporta più protocolli di ingestion: il wire protocol PostgreSQL (PGwire), l’InfluxDB line protocol e un’interfaccia TCP ottimizzata per l’ingest di metriche e telemetria. Questo consente di utilizzare QuestDB come sostituto drop-in per altri database time-series o come backend in architetture esistenti. In aggiunta, l’interfaccia HTTP/REST e la web console integrata forniscono strumenti di interrogazione e visualizzazione immediati, utili durante la fase di sviluppo e debugging.