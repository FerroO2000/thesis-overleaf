\section{Linguaggio Go}
\label{sec:go}

Go è un linguaggio di programmazione compilato, tipizzato staticamente e con gestione automatica della memoria, sviluppato da Google nel 2009 con l'obiettivo di semplificare lo sviluppo di software di sistema concorrente e distribuito su larga scala. La sintassi intenzionalmente essenziale---ispirata a C ma con numerose semplificazioni---favorisce leggibilità e manutenibilità del codice in ambienti industriali~\cite{go:design}. Caratteristiche distintive includono l'assenza di ereditarietà classica, rimpiazzata da composizione e da un modello ad interfacce strutturale più flessibile.

Uno degli elementi più distintivi di Go è il suo modello di concorrenza, basato su \emph{goroutine} e \emph{channel}, che incapsula il principio fondamentale ``do not communicate by sharing memory; share memory by communicating''~\cite{go:concurrency}. Le goroutine sono unità leggere di esecuzione gestite direttamente dal runtime, non mappate uno-a-uno sui thread del sistema operativo: il runtime mantiene internamente uno scheduler che multiplessa migliaia (o milioni) di goroutine su un numero limitato di thread, consentendo di scrivere codice concorrente senza la complessità tradizionale della gestione manuale di thread. I channel sono primitive di comunicazione e sincronizzazione tipizzate che consentono il passaggio di messaggi in modo sicuro tra goroutine, eliminando la necessità di mutex e condizioni di race quando utilizzati correttamente.

Il linguaggio Go fornisce inoltre supporto nativo per strumenti di profiling e osservabilità (runtime statistics, memory profiling, CPU profiling) e include nella libreria standard pacchetti essenziali per reti (\texttt{net}), gestione del tempo (\texttt{time}), operazioni atomiche (\texttt{sync/atomic}) e primitivi di sincronizzazione (\texttt{sync}), costituendo fondamenta robuste per la progettazione di infrastrutture di data processing ad alte prestazioni. La compilazione Go produce un singolo eseguibile binario, senza dipendenze runtime esterne, facilitando il deployment in ambienti containerizzati e cloud-native. Infine, il supporto nativo a test e benchmarking tramite il package \texttt{testing} rende naturale la pratica del test-driven development.