\section[Ingress Stage]{Ingress Stage}

Uno \textit{Ingress stage} rappresenta il punto di ingresso dei dati all'interno della \textit{pipeline} di elaborazione. Costituisce la prima componente che riceve gli input provenienti dall'esterno del sistema e li introduce nel flusso di elaborazione, alimentando così l'intera catena di \textit{stage} successivi. In sostanza, uno \textit{Ingress stage} agisce come una sorgente di dati, traducendo eventi, messaggi o pacchetti provenienti da differenti protocolli o interfacce in un formato gestibile dalla \textit{pipeline} di \textbf{Goccia}.

Nel contesto della libreria, gli \textit{Ingress stage} sono progettati per essere altamente performanti, concorrenti e facilmente integrabili con i meccanismi di comunicazione già definiti dal framework. Ogni \textit{Ingress stage} è quindi responsabile di stabilire una connessione con la sorgente, acquisire i dati in ingresso, trasformarli nel formato previsto e inoltrarli verso lo \textit{stage} successivo.

La libreria \textbf{Goccia} fornisce diversi tipi di \textit{Ingress stage}, ciascuno progettato per scenari operativi o fonti di dati differenti.

\subsection{Ticker}

Lo \textit{Ticker stage} rappresenta l'\textit{Ingress stage} più semplice messo a disposizione dalla libreria \textbf{goccia} ed è concepito per generare in modo autonomo un flusso regolare di messaggi, senza dipendere da sorgenti esterne quali socket di rete, file o code di messaggistica. Dal punto di vista concettuale, si tratta di una sorgente periodica di eventi, particolarmente utile per casi d'uso quali il testing della \textit{pipeline}, il benchmarking di prestazioni, l'iniezione di traffico sintetico o la generazione di "heartbeat" applicativi a intervalli costanti. Ogni "tick" corrisponde alla produzione di un messaggio che attraversa l'intera \textit{pipeline}, permettendo di osservare il comportamento dei vari \textit{stage} a valle in presenza di un carico controllato.

La logica di generazione dei messaggi sfrutta il \texttt{time.Ticker} della libreria standard del Go, il quale mette a disposizione un channel notificato ogni volta che l'intervallo $T$ viene raggiunto, dove $T$ è definito dalla configurazione dello \textit{stage}.

\begin{lstlisting}[language=Go, caption={Ciclo principale del Ticker stage (ingress/ticker.go, metodo run)}]
for {
	// ...
	select {
	case <-ctx.Done():
		return
	case <-ts.ticker.C:
		// generate the ticker message
	}
}
\end{lstlisting}

Nel frammento di codice riportato, è possibile osservare come il costrutto \texttt{select} permetta di ascoltare su molteplici channel in modalità multiplex. In questo caso specifico, il \textit{runtime} rimane in attesa di essere notificato dal ticker oppure della ricezione di un segnale di cancellazione del contesto (\texttt{ctx.Done()}). Quest'ultimo meccanismo è fondamentale per implementare strategie intelligenti di rilascio delle risorse, ad esempio quando il processo riceve un segnale \texttt{SIGINT} o \texttt{SIGTERM}, consentendo così l'implementazione di uno \textit{graceful shutdown} coerente nelle applicazioni. Per questa ragione, nelle librerie Go moderne è prassi richiedere un \texttt{context.Context} come primo argomento di funzioni che impiegano socket o richiedono tempistiche significative per la terminazione.

Il tipo di messaggio prodotto dallo \textit{Ticker stage} è descritto dalla struct \texttt{TickerMessage} e contiene un unico campo specifico, \texttt{TickNumber}, utilizzato per numerare progressivamente i tick. Al momento della generazione del messaggio vengono popolati i campi generici di metadato (tempo di ricezione, timestamp) e viene inizializzata la \textit{root span} per la tracciatura degli attraversamenti degli \textit{stage} successivi, conformemente allo standard OpenTelemetry\cite{otel:trace:api}.

\subsection{UDP}

Lo \textit{UDP stage} rappresenta un \textit{Ingress stage} progettato per ricevere datagrammi \textit{UDP} provenienti dalla rete. A differenza dello \textit{Ticker stage}, che genera autonomamente messaggi a intervalli regolari, lo \textit{UDP stage} rimane in ascolto su un socket di rete e resta in attesa di nuovi datagrammi in arrivo. Questa tipologia di \textit{Ingress stage} risulta particolarmente utile in contesti dove è necessario elaborare flussi di dati provenienti da sorgenti esterne, quali sensori IoT, applicazioni remote, strumenti di monitoraggio di rete o qualsiasi sistema che comunichi tramite il protocollo \textit{UDP}. La natura \textit{connectionless} del protocollo \textit{UDP} consente di ricevere messaggi da molteplici mittenti senza la necessità di stabilire connessioni esplicite, rendendolo ideale per scenari ad alto \textit{throughput} dove la perdita occasionale di datagrammi è accettabile in cambio della bassa latenza.

La libreria \textbf{goccia} mette a disposizione tre parametri di configurazione per lo \textit{UDP stage}: indirizzo IP, porta e dimensione del buffer. I primi due servono a impostare la sorgente da cui ricevere i pacchetti e dispongono di valori di default — \texttt{"0.0.0.0"} per l'indirizzo (che indica l'ascolto su tutte le interfacce di rete) e \texttt{20.000} per la porta. La dimensione del buffer utilizzato per leggere il payload dei datagrammi \textit{UDP} è impostata di default a $1474$ byte, poiché la dimensione massima per un payload Ethernet standard è di $1500$ byte, da cui si sottraggono $28$ byte relativi all'header \textit{UDP}.

Internamente, lo \textit{stage} utilizza un puntatore a una connessione \textit{UDP} fornita dalla libreria standard del Go\cite{go:net:udpconn}. Tale connessione viene inizializzata nel metodo \texttt{Init} del ciclo di vita dello \textit{stage}. Una volta che l'inizializzazione ha esito positivo, è possibile procedere con l'esecuzione dello \textit{stage} tramite il metodo \texttt{Run}. Questo metodo implementa un ciclo \texttt{for} in cui viene richiamato il metodo \texttt{Read} (operazione bloccante) della connessione \textit{UDP}, il quale scrive i byte ricevuti in un buffer pre-allocato alla dimensione configurata. Parallelamente al ciclo principale, viene avviata una goroutine ausiliaria il cui compito è chiudere la connessione \textit{UDP} nel momento in cui viene ricevuta una notifica di cancellazione tramite \texttt{context.Context}. Una volta che la connessione è chiusa, il metodo \texttt{Read} ritorna un errore di tipo \texttt{net.ErrClosed}\cite{go:net:errclosed}, permettendo così al ciclo di uscire in modo ordinato.

\begin{lstlisting}[language=Go, caption={Ciclo principale dello UDP stage (ingress/udp.go, metodo run)}]
go func() {
	<-ctx.Done()
	us.conn.Close()
}()

buf := make([]byte, us.bufferSize)

for {
	// Read the UDP payload
	n, err := us.conn.Read(buf)
	if err != nil {
		// Check if the connection is closed
		if errors.Is(err, net.ErrClosed) {
			// Check if caused by context cancellation
			select {
			case <-ctx.Done():
				return
			default:
			}
		}

		// ...
	}

	// Handle the buffer and send the message ...
}
\end{lstlisting}

Il tipo di messaggio prodotto dallo \textit{UDP stage} è descritto dalla struct \texttt{UDPMessage}, contenente due campi specifici: \texttt{Payload} e \texttt{PayloadSize}. Il messaggio implementa l'interfaccia \texttt{message.Serializable} definita nel package \texttt{message}, consentendo il collegamento di differenti tipi di \textit{Processor stage} a valle che accettano messaggi recanti buffer di byte. Al fine di soddisfare l'interfaccia, è sufficiente implementare il metodo \texttt{GetBytes}, il quale restituisce una slice di byte corrispondente al campo \texttt{Payload} del messaggio.

\begin{lstlisting}[language=Go, caption={Metodo GetBytes di UDPMessage (ingress/udp.go)}]
func (um *UDPMessage) GetBytes() []byte {
	return um.Payload
}
\end{lstlisting}

Poiché il payload è caratterizzato da dimensioni significative, al fine di ridurre il carico sul \textit{garbage collector} attraverso il riuso della memoria e l'eliminazione di continue allocazioni e deallocazioni di migliaia di oggetti, si è optato per l'utilizzo di un \textit{object pool} (\texttt{sync.Pool})\cite{go:sync:pool}. Gli oggetti \texttt{UDPMessage} vengono quindi pre-allocati nel pool e riutilizzati per ogni nuovo datagramma ricevuto, riducendo così la pressione sul sistema di gestione della memoria durante l'elaborazione ad alto throughput.

Lo \textit{stage} espone due metriche destinate al monitoraggio del numero di messaggi e byte ricevuti, implementate tramite l'uso di contatori asincroni\cite{otel:metrics:asynccounter}. Tali metriche possono essere impiegate per monitorare il \textit{throughput} in ingresso alla \textit{pipeline}, fornendo visibilità sulla velocità di ricezione dei dati dalla sorgente di rete.

\subsection{TCP}

Lo \textit{TCP stage} rappresenta un \textit{Ingress stage} progettato per ricevere flussi dati da connessioni \textit{TCP}. A differenza dello \textit{UDP stage}, che gestisce singoli datagrammi provenienti da molteplici mittenti senza stato, il \textit{TCP stage} stabilisce connessioni persistenti con i client e mantiene lo stato della comunicazione per ciascuna connessione. Questa caratteristica lo rende particolarmente adatto per scenari in cui l'affidabilità del trasporto è critica, quali l'acquisizione di log da sistemi remoti, la ricezione di comandi da applicazioni client o l'integrazione con protocolli applicativi che richiedono una comunicazione orientata allo \textit{stream}. Diversamente dallo \textit{Ticker stage} e dallo \textit{UDP stage}, il \textit{TCP stage} introduce una complessità significativamente maggiore, poiché deve gestire molteplici connessioni concorrenti, ciascuna potenzialmente caratterizzata da uno stato diverso, e deve risolvere il problema fondamentale di come delimitare i messaggi all'interno di un flusso di byte continuo.

La libreria \textbf{goccia} mette a disposizione una configurazione flessibile per il \textit{TCP stage}, incapsulata nella struct \texttt{TCPConfig}. I parametri fondamentali sono analoghi a quelli dello \textit{UDP stage}: un indirizzo IP e una porta definiscono il punto di ascolto su cui il server \textit{TCP} rimane in attesa di connessioni in ingresso, mentre un buffer di lettura viene utilizzato per acquisire i dati dalla connessione. Tuttavia, il \textit{TCP stage} introduce ulteriori parametri specifici per la gestione avanzata del flusso.

Il parametro \texttt{ReadTimeout} specifica il tempo massimo di inattività consentito su una connessione prima che essa venga forzatamente chiusa. Questo meccanismo protegge il server da client che si collegano e rimangono silenti indefinitamente, occupando risorse preziose del sistema. Il parametro \texttt{MaxMessageSize} (default 4 MB) definisce la dimensione massima consentita per un messaggio; qualora il buffer accumulato superi questo valore, la connessione viene chiusa al fine di prevenire attacchi di \textit{denial of service} basati su messaggi abnormemente grandi.

Un aspetto cruciale della configurazione è il \textit{framing mode}, definito dal campo \texttt{FramingMode}, il quale specifica come i messaggi vengono delimitati all'interno del flusso \textit{TCP}. Sono supportate due modalità distinte:
\begin{itemize}
    \item \textbf{\textit{Delimited}} (default): I messaggi sono separati da un delimitatore, tipicamente una sequenza di byte come \texttt{"\textbackslash r\textbackslash n"}, impostabile dall'utente nel campo \texttt{Delimiter}. In questa modalità, il \textit{parser} ricerca la sequenza delimitatrice all'interno del buffer accumulato e la utilizza come marcatore di fine messaggio. Esempi di protocolli di livello superiore che impiegano questa strategia sono HTTP/1.x, SMTP e FTP.
    \item \textbf{\textit{Length-Prefixed}}: I messaggi sono preceduti da un header contenente la lunghezza del messaggio. Questa modalità introduce parametri aggiuntivi per estrarre la lunghezza del payload a partire da un header:
    \begin{itemize}
        \item \texttt{HeaderLen} (default 16 byte): la dimensione totale dell'header.
        \item \texttt{MessageLengthFieldOffset} (default 0): l'offset all'interno dell'header dove inizia il campo di lunghezza.
        \item \texttt{MessageLengthFieldLen}: la dimensione del campo di lunghezza (1, 2, 4 o 8 byte).
        \item \texttt{MessageLengthFieldEndianess}: l'ordine dei byte (\textit{little-endian} o \textit{big-endian}) del campo di lunghezza.
    \end{itemize}
\end{itemize}

Infine, il parametro \texttt{OutputQueueSize} specifica la dimensione della coda interna (\textit{fan-in}) che media tra le goroutine dedicate alle connessioni e il \textit{connector} verso lo \textit{stage} successivo.

Il \textit{TCP stage} implementa un'architettura basata su molteplici goroutine sincronizzate. Il ciclo principale si occupa di accettare nuove connessioni tramite il metodo \texttt{Accept} del \textit{listener} \textit{TCP}\cite{go:net:tcplistener:accept} e, per ciascuna nuova connessione in ingresso, avvia una nuova goroutine dedicata alla gestione della relativa comunicazione. In questo modo, si ottiene un pattern di \textit{fan-out}, in cui una singola goroutine principale genera $N$ goroutine worker, una per ogni connessione cliente.

<disegno singola goroutine che genera N goroutine per N connessioni>

Questa architettura è resa possibile dal \textit{runtime} di Go, il quale è in grado di gestire un numero elevato di goroutine e di schedularle efficientemente su un numero finito di thread del sistema operativo.

\begin{lstlisting}[language=Go, caption={Ciclo principale del TCP stage (ingress/tcp.go, metodo run)}]
for {
	// ...

	conn, err := ts.listener.Accept()
	// Check and handle the error ...

	// Spawn a goroutine to handle the connection
	go ts.handleConn(ctx, conn)
}
\end{lstlisting}

Per quanto riguarda la gestione della singola connessione, il flusso è concettualmente simile a quello dello \textit{UDP stage}, con l'eccezione rilevante della divisione del \textit{stream} in messaggi. Tale divisione richiede l'utilizzo di un buffer ausiliario di accumulazione, oltre al buffer di lettura. Il buffer di accumulazione è fondamentale nei casi in cui un messaggio risulti frammentato su più letture: ad esempio, se il buffer di lettura è di 4~KB e il messaggio trasmesso sulla connessione è di 8~KB, saranno necessarie due letture i cui contenuti dovranno essere accumulati sequenzialmente.

\begin{lstlisting}[language=Go, caption={Buffer per la gestione connessione del TCP stage (ingress/tcp.go, metodo handleConn)}]
// Preallocate the accumulator
accBaseCap := min(4*ts.bufferSize, ts.maxMsgSize)
acc := make([]byte, 0, accBaseCap)

// ...

for {
	// ...

	// Set the read deadline
	conn.SetReadDeadline(time.Now().Add(ts.readTimeout))

	// Read the TCP stream
	n, err := conn.Read(buf)
	// Check and handle the error ...

	// Append the new bytes to the accumulator
	acc = append(acc, buf[:n]...)

	// Prevent accumulator from growing too large
	if len(acc) > ts.maxMsgSize {
		ts.tel.LogWarn("message too large, closing connection")
		return
	}

	for {
		// Divide the accumulator into messages ...
	}

	// ...
}
\end{lstlisting}

Dopo l'accumulazione dei byte trasmessi, entra in gioco la logica di divisione dei messaggi secondo la modalità configurata nello \textit{stage} (\textit{delimited} oppure \textit{length-prefixed}). In modalità delimitata, la ricerca della sequenza delimitatrice è effettuata tramite \texttt{bytes.Index}; in modalità \textit{length-prefixed}, il \textit{parser} estrae la lunghezza dal campo di header designato, utilizzando funzioni che gestiscono le diverse combinazioni di lunghezza e \textit{endianess} supportate.

\begin{lstlisting}[language=Go, caption={Divisione messaggi del TCP stage (ingress/tcp.go, metodo handleConn)}]
// ...

for {
	accLen := len(acc)

	// If the accumulator is smaller than the minimum length,
	// continue reading the TCP stream
	if accLen < minAccLen {
		continue loop
	}

	// Get the length of the message.
	msgLen := 0
	totLen := 0
	switch ts.framingMode {
	case TCPFramingModeDelimited:
		// Search for the delimiter
		msgLen = bytes.Index(acc, ts.delimiter)
		totLen = msgLen + ts.delimiterLen

	case TCPFramingModeLengthPrefixed:
		msgLen = ts.parseHeader(acc[:ts.headerLen])
		totLen = msgLen + ts.headerLen
	}

	if msgLen == -1 || accLen < totLen {
		// If the message length is not found or the accumulator is too small,
		// break the loop and continue reading the TCP stream
		break
	}

	// Extract the message
	msg := acc[:totLen]

	// Handle the message and send the result to the output connector ...

	// Remove the message from the accumulator
	acc = acc[totLen:]

	// Check if the accumulator should be reset ...
}

// ...
\end{lstlisting}

Il tipo di messaggio prodotto dal \textit{TCP stage} è descritto dalla struct \texttt{TCPMessage}, la quale implementa l'interfaccia di serializzazione, in modo analogo al messaggio dello \textit{UDP stage}. A differenza di \texttt{UDPMessage}, \texttt{TCPMessage} non utilizza un \textit{object pool}, poiché i messaggi \textit{TCP} possono avere dimensioni molto variabili e l'impiego di un pool con buffer pre-allocato potrebbe risultare inefficiente o insufficiente. Il messaggio porta con sé il payload in byte (\texttt{Message}), la dimensione del payload (\texttt{MessageSize}) e l'indirizzo remoto della connessione (\texttt{RemoteAddr}), fornendo così sia il contenuto informativo sia il contesto di provenienza.

Dal punto di vista dell'osservabilità, il \textit{TCP stage} è per molti aspetti simile allo \textit{UDP stage}. Anche in questo caso sono presenti metriche per il numero totale di messaggi e di byte ricevuti, che consentono di monitorare il \textit{throughput} complessivo della \textit{pipeline}. In aggiunta, viene introdotta una metrica specifica, \texttt{open\_connections}, che traccia il numero di connessioni \textit{TCP} attualmente aperte. Tale metrica offre visibilità sul grado di utilizzo delle risorse del server e permette di individuare con facilità situazioni anomale, come un numero insolitamente elevato di connessioni persistenti o client che non rilasciano correttamente le proprie sessioni.

\subsection{Kafka}

Il \textit{Kafka stage} rappresenta un \textit{Ingress stage} progettato per consumare messaggi da topic \textit{Kafka}, integrandosi direttamente con architetture \textit{event-driven} e sistemi di \textit{streaming} dati. A differenza dei precedenti \textit{Ingress stage} (\textit{Ticker}, \textit{UDP}, \textit{TCP}), il \textit{Kafka stage} non implementa un protocollo di trasporto diretto, ma si affida a un broker centralizzato (\textit{Kafka}) per la gestione persistente e distribuita dei messaggi. Questa caratteristica lo rende ideale in scenari in cui i dati provengono da molteplici produttori asincroni, in cui è necessario garantire una semantica di consegna robusta (\textit{at-least-once}, \textit{exactly-once}), oppure in cui si desiderano costruire pipeline di elaborazione fortemente \textit{decoupled}. La libreria \textbf{Goccia} sfrutta la libreria open-source \textit{segmentio/kafka-go}, un'implementazione Go-nativa del protocollo \textit{Kafka} che evita dipendenze da librerie esterne come \textit{librdkafka}~\cite{segmentio:kafka-go,confluent:librdkafka}.

La configurazione del \textit{Kafka stage} è estremamente flessibile e riflette la complessità intrinseca del sistema \textit{Kafka}. La struct \texttt{KafkaConfig} mima la struttura di configurazione \texttt{ReaderConfig} fornita dalla dipendenza \textit{kafka-go}, così da consentire all'utente una personalizzazione completa della connessione verso i broker \textit{Kafka}~\cite{segmentio:kafka-go:readerconfig}. In questo modo è possibile controllare, tra gli altri, parametri relativi ai broker, al \textit{consumer group}, alle politiche di bilanciamento delle partizioni, ai timeout di lettura e alle strategie di backoff. Di default, lo \textit{stage} consuma i messaggi utilizzando un \textit{group id}, in modo da sfruttare la gestione automatica degli \textit{offset} offerta dal broker quando i dati vengono letti dai topic.

La logica di lettura dello \textit{stage} è volutamente semplice: il cuore dell'implementazione è un unico ciclo \texttt{for} che invoca ripetutamente il metodo \texttt{ReadMessage} del \textit{reader} \textit{kafka-go}~\cite{segmentio:kafka-go:readmessage}. Ogni chiamata restituisce un nuovo messaggio \textit{Kafka}, che viene poi trasformato in una struct \texttt{KafkaMessage} e inserito nel \textit{connector} di output dello \textit{stage}. In caso di errore, se questo è dovuto alla cancellazione del \texttt{context}, il ciclo termina in modo ordinato; negli altri casi l'errore viene registrato tramite il sottosistema di telemetria, e il ciclo procede al tentativo successivo, garantendo una maggiore robustezza in presenza di problemi transitori di rete o di broker.

Dal punto di vista dell'osservabilità, oltre alle metriche comuni a tutti gli \textit{Ingress stage} (numero di messaggi e byte ricevuti), il \textit{Kafka stage} introduce la possibilità di estrarre il contesto di \textit{tracing} direttamente dagli \textit{header} del messaggio \textit{Kafka}. A tale scopo viene utilizzata una struttura dedicata che implementa l'interfaccia \textit{TextMapPropagator} definita dallo standard \textit{OpenTelemetry}~\cite{otel:textmap:propagator}. Questo meccanismo consente di propagare i \textit{trace} tra servizi differenti, permettendo la costruzione di un sistema di \textit{tracing} distribuito in cui ogni messaggio \textit{Kafka} può trasportare il contesto di esecuzione end-to-end lungo l'intera pipeline.

<disegno screenshot trace distribuito>


\subsection{eBPF}

Lo \textit{eBPF stage} rappresenta un \textit{Ingress stage} unico nel panorama della libreria \textbf{Goccia}, in quanto non legge dati da una sorgente di rete tradizionale o da un broker esterno, ma comunica direttamente con il kernel Linux tramite programmi \textit{eBPF} (\textit{extended Berkeley Packet Filter}) che sfruttano una map di tipo \textit{ringbuf}. L'\textit{eBPF} consente l'esecuzione sicura di codice \textit{sandbox} direttamente nel kernel, con \textit{overhead} minimo, abilitando scenari di osservabilità, monitoraggio e \textit{security} altamente efficienti. Lo \textit{eBPF stage} è quindi ideale per la raccolta di eventi dal sistema operativo — quali \textit{syscall}, pacchetti di rete, eventi di file system o segnali di performance — senza richiedere una copia dei dati verso lo spazio utente fino al momento dell'effettiva elaborazione. La libreria \textbf{Goccia} impiega la libreria \textit{open-source} \textit{cilium/ebpf}~\cite{cilium:ebpf}, un \textit{wrapper} Go-nativo che semplifica il caricamento e la gestione di programmi \textit{eBPF} compilati, eliminando la necessità di dipendenze esterne come \textit{libbpf} scritta in C.

L'aspetto più distintivo dello \textit{eBPF stage} è la sua astrazione generica rispetto ai tipi di dati. A differenza degli altri \textit{Ingress stage}, che operano su tipi di messaggio predefiniti, lo \textit{eBPF stage} utilizza i \textit{generics} del Go per permettere agli utenti di definire completamente la struttura dei dati che il programma \textit{eBPF} invia nel \textit{ringbuffer}. Questo approccio consente massima flessibilità: ogni programma \textit{eBPF} produce dati in un formato specifico, e tramite i \textit{generics}, lo \textit{stage} può deserializzare automaticamente i dati grezzi nel tipo Go appropriato.

La configurazione, incapsulata in \texttt{EBPFConfig[O, OPtr]}, si compone di tre componenti funzionali critici:

\begin{itemize}
    \item \textbf{\texttt{LoadFn}}: Una funzione che carica la specifica \textit{eBPF} compilata (generata da \textit{bpf2go}, uno strumento che converte codice \textit{eBPF} scritto in C nelle \textit{bindings} corrispondenti in Go). Questa funzione ritorna una \texttt{ebpf.CollectionSpec}, ossia la rappresentazione in memoria del programma \textit{eBPF} compilato.

    \item \textbf{\texttt{LinkFn}}: Una funzione che ``attacca'' il programma \textit{eBPF} a un punto di \textit{hook} nel kernel. A seconda del tipo di programma \textit{eBPF}, l'\textit{hook} può essere una \textit{syscall}, un'interfaccia di rete (\textit{XDP}), un punto di traccia kernel (\textit{kprobe/uprobe}), oppure altri. Questa funzione ritorna un \texttt{link.Link}~\cite{cilium:ebpf:link}, che rappresenta la connessione attiva tra il programma \textit{eBPF} e il kernel.

    \item \textbf{\texttt{RingBufferGetter}}: Una funzione che estrae la map \textit{eBPF} di tipo \textit{ring buffer} dagli oggetti caricati. Il \textit{ring buffer} rappresenta il meccanismo di comunicazione tra il codice \textit{eBPF} in \textit{kernel space} e l'applicazione in \textit{user space}: il programma \textit{eBPF} scrive dati nel \textit{ring buffer} e lo \textit{stage} legge continuamente da esso.
\end{itemize}

Un parametro opzionale è \texttt{UseUnsafe}, il quale controlla la strategia di deserializzazione dei dati. Se impostato a \texttt{true}, i dati grezzi dal \textit{ring buffer} vengono castati direttamente a struct Go tramite \texttt{unsafe.Pointer}, operazione molto veloce ma che richiede una corrispondenza bit-per-bit tra il layout della struct C del programma \textit{eBPF} e la struct Go. Se impostato a \texttt{false} (default), viene utilizzato \texttt{binary.Read} con descodifica \textit{LittleEndian}, approccio più lento ma robusto e portabile. Opzionalmente, \texttt{CollectionOptions} consente di passare opzioni di caricamento avanzate, quali limiti di memoria, selezioni di programmi specifici, o configurazioni di verifica personalizzate.

L'inizializzazione dello \textit{eBPF stage} è complessa e coinvolge molteplici fasi critiche. Nel metodo \texttt{Init}, la prima operazione è rimuovere i limiti di memoria bloccata tramite \texttt{rlimit.RemoveMemlock}~\cite{cilium:ebpf:rlimit}: i programmi \textit{eBPF} richiedono che determinati buffer (come il \textit{ring buffer}) siano \textit{pinned} in memoria fisica, e il kernel di default limita quanto spazio un processo utente possa bloccare (tipicamente 64 KB). Questa operazione preliminare è essenziale per consentire l'allocazione della memoria necessaria ai programmi \textit{eBPF} senza violare i vincoli di sistema.

Dopo aver rimosso questo limite, la specifica \textit{eBPF} viene caricata tramite \texttt{LoadFn} e i programmi compilati vengono istanziati nel kernel tramite \texttt{spec.LoadAndAssign}. Successivamente, il programma viene attaccato al punto di \textit{hook} tramite \texttt{LinkFn}, attivando così l'esecuzione del codice \textit{eBPF} nel kernel. Infine, la map \textit{ring buffer} viene estratta tramite \texttt{RingBufferGetter} e passata al metodo \texttt{init} della sorgente.

\begin{lstlisting}[language=Go, caption={Inizializzazione del eBPF stage (ingress/ebpf.go, metodo Init)}]
func (es *EBPFStage[T, O, OPtr]) Init(ctx context.Context) error {
	// Remove resource limits for locked memory
	if err := rlimit.RemoveMemlock(); err != nil {
		es.tel.LogError("failed to remove memlock limits", err)
		return err
	}

	// Load the compiled eBPF ELF file
	spec, err := es.cfg.LoadFn()
	if err != nil {
		es.tel.LogError("failed to load eBPF spec", err)
		return err
	}

	// Load the eBPF objects
	var dummyObjs O
	objs := OPtr(&dummyObjs)
	if err := spec.LoadAndAssign(objs, es.cfg.CollectionOptions); err != nil {
		es.tel.LogError("failed to load eBPF objects", err)
		return err
	}
	es.objs = objs

	// Get the link
	link, err := es.cfg.LinkFn(objs)
	if err != nil {
		es.tel.LogError("failed to attach eBPF program", err)
		return err
	}
	es.link = link

	// Get the ring buffer map
	ringBufferMap := es.cfg.RingBufferGetter(objs)

	// ...
}
\end{lstlisting}

Durante la fase di \texttt{Run}, il ciclo principale esegue la lettura dal \textit{ring buffer} tramite il suo metodo \texttt{Read}, che rimane bloccante in attesa di nuovi record. Per ogni record disponibile, viene effettuata la deserializzazione del contenuto nel tipo Go appropriato e il messaggio risultante è inoltrato al \textit{connector} di uscita.

Il tipo di messaggio \texttt{EBPFMessage[T]} è completamente generico e contiene un singolo campo \texttt{Data} di tipo \texttt{T} generico. Questo design semplificato riflette il fatto che i dati \textit{eBPF} sono tipicamente \textit{self-contained}: un evento di \textit{syscall} o un pacchetto di rete contiene tutte le informazioni rilevanti all'interno della struct, senza metadati aggiuntivi da preservare come accade negli altri \textit{Ingress stage}.

Il metodo \texttt{Close} dello \textit{stage} è critico, in quanto deve rilasciare le risorse kernel in modo ordinato e sincronizzato. Innanzitutto, il \textit{ring buffer reader} viene chiuso tramite \texttt{es.rb.Close()}. Successivamente, gli oggetti \textit{eBPF} caricati vengono chiusi, scaricando i programmi dal kernel e liberando le mappe associate. Infine, il \textit{link} viene chiuso tramite \texttt{es.link.Close()}, distaccando il programma \textit{eBPF} dal punto di \textit{hook} kernel. Questa sequenza garantisce che, al termine dello \textit{stage}, tutte le risorse kernel siano state rilasciate e lo stato del kernel sia coerente, prevenendo \textit{resource leak} o comportamenti anomali.

\subsection{File}

Lo \textit{File stage} rappresenta un \textit{Ingress stage} progettato per leggere dati da file all'interno di una o più directory monitorate, offrendo un meccanismo flessibile per l'elaborazione di flussi di dati persistenti. A differenza dei precedenti \textit{Ingress stage}, che consumano dati da sorgenti esterne in tempo reale, lo \textit{File stage} consente di leggere sia file statici preesistenti sia file che vengono creati o modificati dinamicamente durante l'esecuzione della \textit{pipeline}. Questo approccio lo rende ideale per scenari quali l'elaborazione offline di log, il monitoraggio di directory in cui sistemi esterni depositano file di dati, o l'implementazione di \textit{pipeline} di elaborazione \textit{batch} altamente reattive. La libreria \textbf{goccia} sfrutta la libreria \textit{open-source} \textit{fsnotify}\cite{fsnotify:github} per il monitoraggio dei cambiamenti nel file system, consentendo di rilevare automaticamente la creazione, modifica o eliminazione di file.

La configurazione dello \textit{File stage} offre un controllo granulare su come i file vengono letti e processati. Il parametro \texttt{WatchedDirs} specifica l'elenco delle directory da monitorare tramite \textit{fsnotify}. Ogni file presente o creato in tali directory sarà automaticamente sottoposto a lettura.

I parametri di lettura controllano aspetti tecnici dell'acquisizione dei dati. Il parametro fondamentale è \texttt{ChunkSize} (default 4 KB), il quale definisce la dimensione della finestra di lettura dal file tramite il \textit{reader} bufferizzato offerto dalla libreria standard di Go, \texttt{bufio.Reader}\cite{go:bufio:reader}. Tuttavia, lo \textit{File stage} introduce un meccanismo ibrido di delimitazione dei chunk tramite i parametri \texttt{ChunkDelim} (default \texttt{'\textbackslash n'}) e \texttt{MaxChunkSize} (default 32 KB). Per impostazione predefinita, il lettore non si limita a restituire semplici chunk di dimensione fissa, ma continua a leggere oltre il \texttt{ChunkSize} finché non incontra il delimitatore (tipicamente newline), oppure raggiunge la dimensione massima. Questo consente di gestire automaticamente linee di lunghezza variabile senza frammentarle, comportamento particolarmente utile per file di log dove ogni riga rappresenta un record logico indivisibile. Tale comportamento può essere disabilitato, tornando a una lettura sempre di dimensione fissa.

Il parametro \texttt{ForceReRead} (default \texttt{false}) controlla il comportamento in caso di riapertura di un file. Se \texttt{false}, il lettore memorizza l'\textit{offset} dell'ultima lettura e riprende da lì; se \texttt{true}, la lettura riinizia dall'inizio del file, opzione utile per scenari di \textit{replay} o elaborazione idempotente. Il parametro \texttt{CloseDebounce} (default 1 secondo) specifica il tempo di attesa dopo il raggiungimento di EOF prima di chiudere effettivamente il file. Questo meccanismo è fondamentale in scenari in cui il file viene modificato frequentemente: invece di chiudere e riaprire il file ripetutamente, il lettore attende che siano trascorsi $T$ secondi di inattività, riducendo così l'\textit{overhead} di aperture e chiusure successive.

Lo \textit{File stage} implementa un'architettura distribuita basata su molteplici \textit{reader}, uno per ogni file in lettura. Ogni \textit{reader} è eseguito su una goroutine dedicata e comunica tramite una coda condivisa (\textit{fan-in}) con il \textit{bridge}, una goroutine ausiliaria responsabile di inoltrare i messaggi al \textit{connector} di uscita. Questo design consente di leggere parallelamente da molteplici file, mentre la sincronizzazione tramite \textit{fan-in} garantisce un ordinamento coerente dei messaggi verso il resto della \textit{pipeline}.

<disegno stage, reader, fan-in>

Lo \textit{stage} mantiene una mappa di \textit{reader} attivi (\texttt{readers}) e monitora continuamente la directory tramite \texttt{fsnotify.Watcher}\cite{fsnotify:watcher}. Quando \textit{fsnotify} segnala un evento (creazione, modifica, rimozione di file), viene determinata l'azione appropriata: se il file è stato creato o modificato, il lettore è avviato (oppure creato e avviato se non esiste già); se il file è stato eliminato o rinominato, il lettore è chiuso e rimosso dalla mappa. La lettura iniziale dei file preesistenti avviene in \texttt{readExistingFiles}, una routine che scansiona le directory configurate e avvia i \textit{reader} per tutti i file già presenti, poiché \textit{fsnotify} non genera eventi per file che esistevano prima dell'inizio del monitoraggio.

\begin{lstlisting}[language=Go, caption={Ciclo principale del File stage (ingress/file.go, metodo run)}]
// ...

// Before starting the watcher, read all the existing files
fs.readExistingFiles(ctx)

for {
	select {
	case <-ctx.Done():
		return

	case event, ok := <-fs.watcher.Events:
		if !ok {
			return
		}

		// Handle the fsnotify event
		fs.handleEvent(ctx, event)

	case err, ok := <-fs.watcher.Errors:
		if !ok {
			return
		}

		fs.tel.LogError("watcher error", err)
	}
}
\end{lstlisting}

\begin{lstlisting}[language=Go, caption={Gestione eventi fsnotify del File stage (ingress/file.go, metodo handleEvent)}]
func (fs *fileSource) handleEvent(ctx context.Context, event fsnotify.Event) {
	path := event.Name

	// Handle file deletion/renaming
	if event.Op&fsnotify.Remove == fsnotify.Remove ||
		event.Op&fsnotify.Rename == fsnotify.Rename {

		if fs.hasReader(path) {
			fs.removeReader(path)
		}

		return
	}

	// Handle file creation
	if event.Op&fsnotify.Create == fsnotify.Create {
		if fs.hasReader(path) {
			fs.startReader(ctx, path)
		} else {
			fs.addAndStartReader(ctx, path)
		}

		return
	}

	// Handle file modification
	if event.Op&fsnotify.Write == fsnotify.Write {
		if fs.hasReader(path) {
			fs.startReader(ctx, path)
		} else {
			fs.addAndStartReader(ctx, path)
		}

		return
	}
}
\end{lstlisting}

Ogni \textit{reader} mantiene una macchina a stati con quattro stati principali: \texttt{Idle} (non ancora avviato), \texttt{Started} (in lettura attiva), \texttt{Paused} (in attesa dopo EOF), e \texttt{Closed} (chiuso e liberato). Il ciclo di lettura del \textit{reader} legge chunk sequenziali dal file. Quando la ricerca del delimitatore è abilitata, il lettore applica la seguente logica: dopo ogni lettura, verifica se l'ultimo byte del chunk è il delimitatore; se non lo è, continua a leggere byte aggiuntivi finché non trova il delimitatore o raggiunge \texttt{MaxChunkSize}. Questo approccio ibrido combina l'efficienza della lettura bufferizzata con la correttezza logica delle linee complete.

Quando EOF è raggiunto, il lettore entra nello stato \texttt{Paused} e attende per \texttt{CloseDebounce}. Se durante questo periodo il file viene modificato e \textit{fsnotify} invia un evento di scrittura, il lettore è riavviato; altrimenti, dopo il \textit{timeout}, il file è chiuso e il \textit{reader} terminato. Questo meccanismo di \textit{debounce} riduce drasticamente la frequenza di aperture e chiusure, specialmente in scenari in cui file di log vengono scritti incrementalmente.

Anche il messaggio dello \textit{File stage} implementa l'interfaccia di serializzazione. La struct \texttt{FileMessage} contiene metadati specifici per i file: \texttt{Path} (il percorso del file), \texttt{Chunk} (il buffer contenente i dati letti), \texttt{ChunkSize} (la dimensione effettiva del chunk), \texttt{Offset} (l'\textit{offset} nel file prima che il chunk fosse letto), e \texttt{DelimiterFound} (un booleano che indica se il chunk termina con il delimitatore configurato). Tali metadati consentono ai \textit{Processor stage} a valle di accedere sia ai dati sia alle informazioni di contesto, abilitando scenari sofisticati quali la rielaborazione selettiva di range di file o il debug basato sulla posizione.

Lo \textit{File stage} espone tre metriche: \texttt{readers} (numero totale di \textit{reader} gestiti), \texttt{active\_readers} (numero di \textit{reader} in lettura attiva), e \texttt{read\_bytes} (numero totale di byte letti). Inoltre, ogni chunk è accompagnato da uno span OpenTelemetry che registra la dimensione del chunk, la presenza di dati letti oltre il \texttt{ChunkSize}, e il percorso del file, fornendo visibilità completa sulla performance di lettura.

