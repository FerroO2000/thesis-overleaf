\section[Egress Stage]{Egress Stage}

Un Egress stage rappresenta il punto terminale della pipeline di \textbf{Goccia}, responsabile dell'esportazione dei messaggi elaborati verso destinazioni esterne al sistema. Diversamente dai Processor stage, che trasformano messaggi e li inoltrano a stage successivi, gli Egress stage ricevono messaggi da uno stage precedente tramite un connector di input e li materializzano verso sink esterni quali database, broker di messaggistica, file system o socket di rete. In altri termini, ogni Egress stage implementa una funzione di output: \(f(\text{messaggio\_in}) \rightarrow \text{void}\), terminando così il flusso di elaborazione e rendendo i dati disponibili per sistemi downstream o per la persistenza.

\subsection{UDP Egress}

Lo \textit{UDP stage} è un \textit{Egress stage} specializzato nell'invio di messaggi verso un endpoint remoto tramite il protocollo \textit{UDP}. Contrariamente allo \textit{UDP Ingress stage}, che riceve dati grezzi da una socket \textit{UDP}, l'\textit{UDP Egress stage} conclude la \textit{pipeline} trasmettendo messaggi elaborati a una destinazione esterna, fungendo da punto di uscita per il flusso di dati.

La configurazione dello \textit{stage} è minimale: richiede semplicemente l'indirizzo IP di destinazione (con \textit{default} 127.0.0.1) e la porta di destinazione (\textit{default} 20000).

L'\textit{UDP Egress stage} può ricevere messaggi serializzati da qualunque \textit{Processor stage} che produca output serializzabile: dati \textit{CSV}, messaggi \textit{CAN} decodificati, output di elaborazioni personalizzate, eccetera. Non vi è accoppiamento di tipo; l'unico vincolo è che il messaggio implementi l'interfaccia \texttt{Serializable}.

Il cuore dello \textit{stage} è il suo \textit{worker}, il quale si occupa di estrarre il payload serializzato dal messaggio tramite \texttt{GetBytes()} e di trasmetterlo sul socket \textit{UDP} mediante \texttt{conn.Write}~\cite{go:net:udpconn:write}.

Se l'operazione di scrittura fallisce (ad esempio, per perdita di connessione, \textit{timeout} di rete, o satura del \textit{buffer} del kernel), il \textit{worker} registra l'errore ma non interrompe l'elaborazione: il messaggio è semplicemente marcato come fallito nella metrica \texttt{delivering\_errors}. Questo riflette la natura \textit{best-effort} di \textit{UDP}: non vi è garanzia di consegna, ritrasmissione automatica, o conferma di ricezione. È responsabilità dello \textit{stage} o dell'applicazione a monte gestire affidabilità se richiesta (ad esempio, tramite ACK a livello applicativo o ricezione esplicita).

Come altri \textit{Egress/Processor stage}, l'\textit{UDP stage} supporta sia \textit{Single Worker Mode} che \textit{Worker Pool Mode}. In modalità \textit{pool}, il \textit{framework} gestisce automaticamente lo \textit{scaling} dei \textit{worker} in base al carico, usando la stessa connessione \textit{UDP} condivisa tra tutti i \textit{worker}. Questo consente di sfruttare il parallelismo \textit{multicore} per aumentare il \textit{throughput} di trasmissione, purché il kernel e l'hardware di rete lo supportino.

In sintesi, l'\textit{UDP Egress stage} incarna il paradigma di esportazione minimalista e ad alte prestazioni: prende un flusso di messaggi elaborati e li spedisce direttamente a una destinazione esterna usando il protocollo \textit{UDP}, senza \textit{buffering} persistente, rielaborazione, o garanzie di consegna. È ideale per scenari dove la velocità è prioritaria rispetto all'affidabilità, come telemetria \textit{real-time}, \textit{streaming} di dati, o distribuzione di messaggi a sistemi esterni a bassa latenza.

\subsection{TCP Egress}

Lo \textit{TCP Egress stage} è un \textit{Egress stage} specializzato nell'invio di messaggi verso un endpoint remoto tramite una connessione \textit{TCP} persistente. A differenza dello \textit{UDP Egress stage}, che invia datagrammi isolati senza stato, il \textit{TCP Egress stage} mantiene una connessione bidirezionale affidabile con il destinatario, garantendo consegna ordinata e completa di tutti i dati.

La configurazione dello \textit{TCP Egress stage} è simile a quella dello \textit{UDP Egress stage}, ma con un parametro aggiuntivo: \texttt{WriteTimeout}, che specifica il tempo massimo di attesa per una singola operazione di scrittura (\textit{default} 10 secondi). Questo timeout protegge da situazioni di deadlock dove la connessione è ancora aperta ma il destinatario non legge dati (ad esempio, a causa di crash o sovraccarico).

Contrariamente a \textit{UDP} (\textit{best-effort}), \textit{TCP} garantisce che ogni byte trasmesso arriverà al destinatario nell'ordine esatto, oppure che un errore sia riportato. Se la connessione si interrompe o il destinatario non legge i dati entro il timeout configurato, il \textit{worker} registra un errore e la metrica \texttt{delivering\_errors} è incrementata.

Il \textit{worker} dello \textit{stage} estrae il payload serializzato dal messaggio in input e lo trasmette tramite \texttt{conn.Write}\cite{go:net:tcpconn:write} come sequenza continua di byte, senza delimitatori impliciti: se il messaggio necessita di delimitazione (ad esempio, newline), deve essere già incluso nel payload serializzato dallo \textit{stage} precedente.

Il \textit{TCP Egress stage} supporta solamente la \textit{Single Worker Mode}, in quanto, sebbene il metodo \texttt{conn.Write} possa essere chiamato da goroutine differenti in maniera sicura, il contenuto dei dati trasmessi potrebbe intervallarsi. Per esempio, se una goroutine scrivesse ``hello'' e un'altra ``word'', il dato trasmesso potrebbe essere una sequenza del tipo ``heworldllo''.

Lo \textit{TCP Egress stage} è ideale per scenari dove l'affidabilità è critica: trasmissione di comandi verso sistemi embedded, esportazione di dati strutturati verso data warehouse, o comunicazione con sistemi remoti che richiedono ricezione garantita.

In sintesi, il \textit{TCP Egress stage} rappresenta l'alternativa affidabile allo \textit{UDP Egress stage}: conclude la \textit{pipeline} trasmettendo messaggi elaborati tramite una connessione \textit{TCP} persistente, garantendo consegna ordinata e completa, con timeout configurabile per proteggere da blocchi indefiniti.

\subsection{Kafka Egress Stage}

Lo \textit{Kafka Egress stage} è uno \textit{Egress stage} specializzato nell'invio di messaggi verso un broker \textit{Apache Kafka}. Lo stage conclude la pipeline inviando messaggi a topic \textit{Kafka} con semantiche di consegna affidabili e configurabili, fungendo da punto di uscita per il flusso di dati verso un sistema \textit{event-driven} distribuito.

La configurazione dello \textit{Kafka Egress stage} riprende i campi definiti dalla libreria \textit{kafka-go} nella sua struttura \texttt{kafka.Writer}~\cite{segmentio:kafka-go:writer}, in maniera speculare a come viene fatto nell'omonimo stage di \textit{Ingress}. Il parametro fondamentale resta quello del campo \texttt{Brokers}, che definisce gli endpoint a cui connettersi. Contrariamente allo stage di \textit{Ingress}, i topic sono definiti all'interno del messaggio, permettendo di scegliere a \textit{runtime} dove indirizzare il messaggio.

La logica dello stage è minimale, in quanto viene semplicemente preso il messaggio in ingresso, viene creata la struttura usata per contenere le informazioni riguardanti gli header, la chiave, il \texttt{value} (payload) e il topic in cui il messaggio deve essere inoltrato. Infine, viene richiamata la funzione \texttt{WriteMessages}~\cite{segmentio:kafka-go:writemessages} che si occuperà di passare il dato al broker \textit{Kafka}.

Come altri \textit{Egress stage}, il \textit{Kafka Egress stage} supporta sia \textit{Single Worker Mode} che \textit{Worker Pool Mode}, poiché il writer è \textit{thread-safe} e gestisce internamente la serializzazione e il batching dei messaggi.

Un aspetto distintivo del \textit{Kafka Egress stage} è l'integrazione nativa del tracciamento distribuito. Ogni messaggio inviato include nel suo header \textit{Kafka} il contesto di trace (trace ID, span ID, baggage) estratto dal messaggio in transito, consentendo la propagazione del contesto di esecuzione end-to-end attraverso il cluster \textit{Kafka} verso sistemi consumer a valle, abilitando così la correlazione completa dei flussi di dati in architetture microservizi.

\subsection{File Egress Stage}

Lo \textit{File Egress stage} rappresenta uno stage di esportazione progettato per persistere messaggi elaborati dalla pipeline in un file su disco, funzionando come punto di uscita persistente per flussi di dati che richiedono archiviazione locale o condivisione tramite \textit{filesystem}. Esso fornisce un meccanismo semplice e deterministico per scrivere sequenzialmente messaggi su \textit{filesystem} locale (append only), ideale per logging, tracing, analisi offline, e backup di dati elaborati.

La libreria \textbf{Goccia} fornisce una configurazione minimalista per il \textit{File Egress stage}, che mira a controllare i comportamenti di buffering e flushing dei dati verso il disco. Il parametro principale è \texttt{Path}, ovvero il percorso del file di destinazione, e la dimensione del buffer in cui accumulare i byte prima della scrittura su disco (default 4096 byte). Inoltre, vi sono altri due parametri per regolare il flushing, utili per forzare una scrittura del file a determinate condizioni. Quest'ultime riguardano la percentuale di riempimento del buffer e una deadline temporale (default: 1 secondo). La configurazione del buffering consente di controllare il trade-off tra latenza e throughput: buffer di piccole dimensioni e threshold basse producono flush frequenti, incrementando il numero di operazioni I/O su disco; buffer più grandi e threshold elevate riducono il numero di operazioni I/O ma introducono latenza maggiore nella persistenza.

Lo stage adotta un'architettura \textit{Single Worker Mode}, in contrasto con altri stage che supportano \textit{worker pool}. Questa scelta è dovuta alla natura sequenziale e seriale della scrittura su file: accedere contemporaneamente allo stesso file descriptor tramite molteplici goroutine richiederebbe sincronizzazione aggiuntiva e potrebbe compromettere l'ordine di scrittura dei messaggi. Mantenere un singolo worker garantisce che i messaggi siano scritti nell'ordine esatto di ricezione dalla pipeline, preservando la causalità del flusso dati. Nel caso in cui si voglia scrivere più di un file, è richiesto l'uso di \textit{N} stage di questo tipo, quanti sono i file.

Il messaggio in ingresso allo stage è generico, ma deve implementare l'interfaccia \texttt{message.Serializable}, come molti altri stage della libreria. Per ciascun messaggio ricevuto, viene estratto il payload del messaggio, che poi è scritto nel \texttt{bufio.Writer}~\cite{golang:bufio:writer}. Questa operazione non scrive immediatamente su disco, ma accumula i byte nel buffer in memoria. Il numero di byte scritti è registrato per tracciamento. Successivamente, viene valutata la soglia di flush: se il numero di byte accumulati supera la soglia configurata, il metodo di flush viene invocato per trasmettere immediatamente i dati dal buffer al kernel. Parallelamente al meccanismo di flush basato su soglia, una goroutine ausiliaria rimane in ascolto su un ticker, e ogni volta che l'intervallo di deadline configurata scade, invoca il flush.

Lo stage \textit{File Egress} espone tre metriche per il monitoraggio: la prima conta il numero dei byte scritti, la seconda il numero di errori durante l'operazione di scrittura nel buffer, mentre l'ultima conta gli errori di flushing.

Il \textit{File Egress stage} è ideale per scenari in cui la persistenza locale è sufficiente e la semplicità è prioritaria rispetto alla distribuzione. Un esempio pratico è il logging.

In sintesi, il \textit{File Egress stage} incarna il paradigma di persistenza locale e sequenziale: prende un flusso di messaggi elaborati dalla pipeline e li scrive ordinatamente in un file su disco, offrendo controllo granulare sul buffering tramite soglie dinamiche e flushing periodico, preservazione rigorosa dell'ordine causale, e sincronizzazione robusta verso il \textit{filesystem}. È il componente ideale per conclusioni di pipeline che richiedono archiviazione durabile, facilità di analisi offline, e semplicità operativa.

\subsection{QuestDB Egress Stage}

Lo \textit{QuestDB Egress stage} rappresenta uno stage di esportazione progettato per persistere messaggi elaborati verso \textit{QuestDB}~\cite{questdb:official}, un database time-series colonnare ad alte prestazioni.

La configurazione dello stage presenta un unico campo specifico, ovvero l'indirizzo di riferimento del database. La semplicità della configurazione rispecchia l'approccio di \textit{QuestDB}: la libreria client \textit{go-questdb-client}~\cite{questdb:go-client} gestisce internamente aspetti quali il pooling delle connessioni, il flushing automatico dei messaggi, e la ricongiunzione in caso di fallimento.

Lo stage riceve messaggi del tipo \texttt{QuestDBMessage}, che incapsula una collezione di righe (rows) da inserire. Ciascuna riga è associata a una tabella specifica e contiene simboli (symbol columns) e colonne (value columns). I simboli rappresentano colonne di tipo categorico in \textit{QuestDB}, ottimizzate per l'indicizzazione e la deduplicazione: ogni simbolo deve essere inserito prima di qualunque altra colonna della riga. Le colonne rappresentano dati di tipo vario: booleano, intero, float, stringa, timestamp, e long integer (\textit{big.Int}).

Lo stage, per ciascun messaggio ricevuto, elabora tutte le righe contenute tramite un iteratore. Per ogni riga, seleziona la giusta tabella, poi inserisce tutti i simboli prima di inserire i valori delle colonne. Una volta caricati tutti i valori delle colonne, viene impostato il timestamp della riga.

Lo stage supporta \textit{Worker Pool Mode} (oltre a \textit{Single Worker Mode}). In modalità pool, molteplici worker condividono lo stesso \texttt{LineSenderPool}~\cite{questdb:go-client:linesenderpool}, dal quale ciascun worker estrae un sender indipendente. Poiché il pool garantisce \textit{thread-safety}, il parallelismo multicore è sfruttato per incrementare il throughput di inserimento verso \textit{QuestDB}. Il pool gestisce automaticamente il flushing dei buffer quando la soglia di autoflush è raggiunta, coordinando gli inserimenti provenienti da molteplici worker in modo trasparente. Se il pool rileva fallimenti di connessione, applica retry automatici con timeout configurato, garantendo resilienza a livello di trasporto.

Lo stage espone una singola metrica di monitoraggio che tiene traccia del numero di righe inserite.

Il \textit{QuestDB Egress stage} è ideale per scenari dove la persistenza strutturata in un database time-series è prioritaria. Un esempio classico di tale utilizzo è l'implementazione di un sistema di telemetria \textit{real-time}, per inviare metriche, trace e log strutturati verso \textit{QuestDB} per analisi e visualizzazione su dashboard. Questo è anche il principale caso d'utilizzo che ha portato alla creazione dello stesso progetto \textbf{Goccia}.

In sintesi, il \textit{QuestDB Egress stage} incarna il paradigma di persistenza strutturata e ad alte prestazioni: prende un flusso di messaggi elaborati e li inserisce in una banca dati time-series, offrendo tipizzazione di dati, pooling di connessioni, flushing automatico, e supporto a molteplici tabelle e simboli categorici. È il componente ideale per conclusioni di pipeline che richiedono persistenza analitica, query \textit{real-time}, e integrazione con ecosistemi di \textit{business intelligence} e monitoraggio.

\subsection{Sink Egress Stage}

Lo \textit{Sink Egress stage} rappresenta il caso limite di uno stage di esportazione: non persiste, non trasmette, non scrive alcun dato verso destinazioni esterne. Invece, consuma semplicemente tutti i messaggi in ingresso, li distrugge, e prosegue. È concepito esclusivamente per scopi di testing e benchmarking, permettendo di valutare le prestazioni della pipeline senza l'overhead di I/O verso \textit{filesystem}, rete, o database.

