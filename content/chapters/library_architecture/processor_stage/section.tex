\section[Processor Stage]{Processor Stage}

Un \textit{Processor stage} rappresenta un'unità di elaborazione intermedia all'interno della \textit{pipeline} di \textbf{goccia}. Diversamente dagli \textit{Ingress stage}, che fungono da punto di ingresso, i \textit{Processor stage} ricevono messaggi da uno \textit{stage} precedente tramite un \textit{connector} di input, li trasformano secondo una logica specifica, e inoltrano il risultato verso lo \textit{stage} successivo tramite un \textit{connector} di output. In altri termini, ogni \textit{Processor stage} implementa una funzione di trasformazione pura: $f(\text{messaggio\_in}) \to \text{messaggio\_out}$, consentendo di costruire \textit{pipeline} complesse attraverso la composizione di trasformazioni modulari. Nel contesto della libreria \textbf{goccia}, i \textit{Processor stage} sono progettati per essere altamente flessibili e riusabili, mantenendo una chiara separazione tra l'infrastruttura di orchestrazione e la logica di elaborazione specifica.

\subsection{Cannelloni}

Lo \textit{Cannelloni stage} rappresenta un \textit{Processor stage} specializzato per la serializzazione e deserializzazione di messaggi nel formato \textit{Cannelloni}\cite{cannelloni:github}, uno standard di incapsulamento per messaggi \textit{CAN} (\textit{Controller Area Network}) progettato per il trasporto su reti IP. La libreria \textbf{goccia} fornisce due varianti dello \textit{stage}: il \textit{Cannelloni Decoder stage}, che trasforma dati \textit{Cannelloni} grezzi in messaggi \textit{CAN} strutturati, e il \textit{Cannelloni Encoder stage}, che compie l'operazione inversa. Sebbene logicamente rappresentino un unico tipo di elaborazione (bidirezionale), sono implementati come due \textit{stage} separati per conformarsi al modello unidirezionale di trasformazione della \textit{pipeline}.

Il protocollo \textit{CAN} è ampiamente utilizzato in ambito automotive e nei sistemi embedded per la comunicazione \textit{real-time} a bassa latenza. Tuttavia, il \textit{CAN} tradizionale è limitato a reti locali (bus seriale). Il formato \textit{Cannelloni} risolve questo vincolo consentendo l'incapsulamento di molteplici messaggi \textit{CAN} in un singolo pacchetto \textit{UDP}, abilitando il trasporto su reti IP. Questo approccio è particolarmente rilevante in scenari dove i dati \textit{CAN} provengono da \textit{gateway} remoti e devono essere elaborati centralmente.

<disegno frame cannelloni>

Il formato \textit{Cannelloni} è organizzato gerarchicamente. L'header del frame è composto da 5 byte: il campo \texttt{version} (1 byte) specifica la versione del protocollo; \texttt{opCode} (1 byte) contiene il codice operativo; \texttt{sequenceNumber} (1 byte) è il numero sequenziale del frame, utilizzato per il riordino e la rilevazione di perdite; \texttt{messageCount} (2 byte, \textit{big-endian}) indica il numero di messaggi \textit{CAN} contenuti nel frame. Segue quindi la sequenza dei messaggi \textit{CAN}, ciascuno con lunghezza variabile (minimo 5 byte): \texttt{canID} (4 byte, \textit{big-endian}) è l'identificatore del messaggio; \texttt{dataLen} (1 byte) specifica la lunghezza dei dati (0-8 per \textit{CAN} 2.0, 0-64 per \textit{CAN} FD), con il bit più significativo (0x80) che indica se il messaggio è \textit{CAN} FD; \texttt{canFDFlags} (1 byte, opzionale) contiene flag specifici di \textit{CAN} FD, presenti solamente se il bit di \textit{CAN} FD è impostato; \texttt{data} (variabile) è il payload del messaggio.

Il \textit{Cannelloni Decoder stage} riceve dati \textit{Cannelloni} grezzi (tipicamente da uno \textit{UDP stage}) e li decodifica in messaggi \textit{CAN} strutturati, eseguendo il parsing sequenziale del buffer contenente i dati grezzi. Questo buffer è ottenuto richiamando il metodo \texttt{GetBytes} definito dall'interfaccia di serializzazione utilizzata dagli \textit{Ingress stage} quali \textit{UDP}, \textit{TCP}, \textit{File} e \textit{Kafka}. Il \textit{decoder} verifica innanzitutto che il buffer contenga almeno 5 byte per l'header, quindi estrae i campi utilizzando accesso diretto ai byte e \texttt{binary.BigEndian} per i campi multi-byte. Per ogni messaggio \textit{CAN} nel frame, procede sequenzialmente estraendo l'identificatore, la lunghezza dei dati, e verificando se il messaggio è di tipo \textit{CAN} FD, per concludere copiando il payload nel messaggio deserializzato.

Il \textit{Cannelloni Encoder stage} compie l'operazione inversa: riceve messaggi di tipo \texttt{CannelloniMessage} (tipicamente generati da uno \textit{stage} precedente o costruiti manualmente) e li serializza in \textit{buffer} di byte. Questo \textit{buffer} viene successivamente inserito nel campo apposito del tipo \texttt{CannelloniEncodedMessage}.

Il messaggio \texttt{CannelloniMessage} implementa l'interfaccia \texttt{message.ReOrderable}, la quale espone il metodo \texttt{GetSequenceNumber}. Questo consente allo \textit{stage} \textit{ROB} (\textit{ReOrder Buffer}) posizionato a valle del decoder di riordinare i frame in arrivo in caso di perdita di pacchetti o arrivo fuori ordine dalla rete. Il numero sequenziale è mantenuto durante la codifica, garantendo che il riordino sia semanticamente corretto.

Sia il decoder che l'encoder generano uno span \textit{OpenTelemetry} per ogni frame elaborato, registrando il numero di messaggi \textit{CAN} contenuti nel frame tramite l'attributo \texttt{message\_count}. Questo fornisce visibilità granulare sui volumi di dati elaborati e sulla struttura interna dei frame, facilitando il monitoraggio della salute della \textit{pipeline} e la diagnostica di anomalie.

\subsection{CAN}

Il \textit{CAN stage} rappresenta un \textit{Processor stage} specializzato per la decodifica di messaggi \textit{CAN} (\textit{Controller Area Network}) da un formato grezzo a una rappresentazione strutturata di segnali decodificati. A differenza dello \textit{Cannelloni stage}, che gestisce il livello di trasporto e incapsulamento su reti IP, il \textit{CAN stage} opera al livello semantico, estraendo i singoli segnali contenuti nei messaggi \textit{CAN} secondo una definizione di schema.

Un messaggio \textit{CAN} grezzo è semplicemente un identificatore (\textit{CAN ID}) associato a una serie di byte (payload). Tuttavia, i byte non sono auto-descrittivi: la loro interpretazione dipende completamente dallo schema di decodifica definito dai progettisti del sistema. Ad esempio, un payload di 8 byte potrebbe contenere pressione, temperatura, stato di allarme e altre grandezze fisiche, ognuna occupando un intervallo di bit specifico all'interno del buffer. Questi campi contenuti nel payload prendono il nome di \textit{segnali}. Lo \textit{CAN stage} traduce questa rappresentazione binaria in segnali nominati con valori tipizzati (booleani, interi, numeri in virgola mobile, enumerazioni), rendendo i dati direttamente intelligibili alle applicazioni a valle della \textit{pipeline}.

La libreria \textbf{Goccia} integra la libreria \textit{acmelib}~\cite{acmelib:github}, un \textit{framework} Go per la definizione e manipolazione di schemi \textit{CAN}. La configurazione dello \textit{CAN stage} accetta un elenco di oggetti \texttt{acmelib.Message}, ognuno dei quali rappresenta il modello di un messaggio \textit{CAN} specifico (identificato da un \textit{CAN ID}) e contiene la definizione del layout dei segnali.

All'inizializzazione dello \textit{stage}, viene costruita una mappa che associa a ogni \textit{CAN ID} la funzione di decodifica corrispondente. Durante l'elaborazione, quando arriva un messaggio grezzo, il \textit{decoder} recupera la funzione dalla mappa e l'invoca sul payload, ottenendo una lista di segnali decodificati con nome, valore grezzo e valore tipizzato.

Il messaggio ritornato dallo \textit{stage} incorpora la lista dei segnali decodificati. Per ogni segnale viene utilizzato un approccio \textit{tagging} tramite il campo \texttt{Type} per indicarne il tipo (booleano, intero, numero in virgola mobile, enumerazione), e il valore effettivo è memorizzato nel corrispondente campo tipizzato (\texttt{ValueFlag}, \texttt{ValueInt}, \texttt{ValueFloat}, \texttt{ValueEnum}). Questo approccio consente una gestione \textit{type-safe} senza ricorrere a interfacce generiche o \textit{reflection}, a scapito di un maggiore utilizzo di memoria.
\subsection{CSV}

Lo \textit{CSV stage} rappresenta un \textit{Processor stage} specializzato per la serializzazione e deserializzazione di dati in formato \textit{CSV} (\textit{Comma-Separated Values}). Come lo \textit{Cannelloni stage}, la libreria \textbf{Goccia} fornisce due varianti: il \textit{CSV Decoder stage}, che trasforma dati \textit{CSV} grezzi in messaggi strutturati, e il \textit{CSV Encoder stage}, che compie l'operazione inversa. Questo consente l'integrazione di sorgenti dati \textit{CSV} (file, stream di rete) all'interno della \textit{pipeline} e l'esportazione di risultati elaborati in formato \textit{CSV}.

La configurazione dello \textit{CSV stage} è basata su uno schema dichiarativo composto dalle definizioni delle colonne. Per ogni colonna viene specificato il nome, il tipo di dato (stringa, intero, numero in virgola mobile, booleano, \textit{timestamp}) e, per \textit{timestamp}, il layout di \textit{parsing} (ad esempio RFC3339, ISO 8601, o layout personalizzato). Questo approccio consente di gestire file \textit{CSV} con tipi di dati eterogenei senza perdere informazioni di tipo durante la deserializzazione.

Il \textit{CSV Decoder stage} riceve dati \textit{CSV} grezzi (tipicamente da uno \textit{File stage} o \textit{UDP stage}) e li decodifica in messaggi strutturati. L'algoritmo di decodifica scansiona sequenzialmente il buffer di dati grezzi, carattere per carattere, accumulando i dati di una colonna in un buffer di stringhe. Quando incontra un byte che fa parte di un carattere multi-byte UTF-8 (bit più significativo impostato), utilizza \texttt{utf8.DecodeRune}~\cite{go:utf8:decoderune} per estrarre correttamente il carattere \textit{Unicode}, garantendo la compatibilità con dataset internazionali. I delimitatori di colonna (virgola) e di riga (\textit{newline}) segnalano il completamento di un valore e l'inizio di un nuovo. Il decoder gestisce correttamente righe incomplete, \textit{newline} assenti all'ultimo valore, e \textit{carriage return} (\texttt{\textbackslash r}) (comune nei file \textit{CSV} esportati da sistemi Windows). Per ogni colonna completata, il decoder estrae il tipo dalla definizione dello schema e invoca il metodo di decodifica specializzato sfruttando il pacchetto \texttt{strconv} della libreria standard~\cite{go:strconv}. Se la conversione del tipo fallisce (ad esempio, una stringa non convertibile a intero), nel messaggio generato dallo \textit{stage} viene impostato a \texttt{false} un flag che segnala l'insuccesso della decodifica. Questo approccio di \textit{soft-validation} è utile in scenari dove dati sporchi sono comuni.

Il \textit{CSV Encoder stage} compie l'operazione inversa: riceve messaggi \texttt{CSVMessage} (liste di righe) e li serializza in buffer \textit{CSV} grezzi. L'encoder, per ogni riga e colonna, scrive il valore tipizzato nello stream testuale, utilizzando le funzioni \texttt{strconv} per convertire interi e float a stringhe. Nel caso in cui una colonna non dovesse contenere un dato valido, l'encoder provvede a scrivere un valore di \textit{default}, garantendo che il chunk di file \textit{CSV} risultante sia sempre ben formattato, anche con dati incompleti. L'encoder utilizza un \texttt{strings.Builder} con capacità pre-allocata per ridurre le allocazioni dinamiche, massimizzando l'efficienza della serializzazione.

Un aspetto notevole è l'uso di \textit{object pooling} per i messaggi \texttt{CSVMessage}, in maniera analoga ai messaggi risultanti dallo \textit{UDP Ingress stage}. Viene utilizzato un \texttt{sync.Pool} per riciclare le istanze dei messaggi, riducendo la pressione sul \textit{garbage collector}. Quando un messaggio è distrutto, viene ripulito e restituito al pool per essere riutilizzato nella prossima decodifica. Questo è particolarmente importante in scenari con alto \textit{throughput} dove decine di migliaia di messaggi \textit{CSV} vengono processati al secondo. Similarmente agli altri \textit{Processor stage}, il \textit{CSV Decoder stage} è generico rispetto al tipo di input, consentendo di accettare dati \textit{CSV} da qualunque \textit{stage} che ritorni un messaggio implementante l'interfaccia di serializzazione. Il flusso tipico potrebbe essere: \textit{File stage} $\to$ \textit{CSV Decoder stage} $\to$ elaborazione specifica del dominio, oppure il percorso inverso per l'esportazione.

\subsection{Custom}

Lo \textit{Custom stage} è un \textit{Processor stage} straordinariamente flessibile che consente agli utenti di implementare logica di elaborazione arbitraria tramite l'implementazione di un'interfaccia personalizzata. A differenza degli \textit{stage} predefiniti (\textit{Filter}, \textit{CAN}, \textit{CSV}), che implementano trasformazioni specifiche del dominio, il \textit{Custom stage} fornisce un'astrazione generica che delega completamente la logica di \textit{processing} all'utente, consentendo di elaborare qualunque tipo di messaggio e produrre qualunque tipo di output.

Il cuore dello \textit{Custom stage} è l'interfaccia \texttt{CustomHandler}, che richiede di implementare quattro metodi:

\begin{itemize}
    \item \textbf{\texttt{Init}}: Invocato una sola volta al momento dell'inizializzazione dello \textit{stage}, prima che qualunque messaggio sia processato. Utile per inizializzare risorse, connessioni, cache, o stato interno specifico dell'applicazione.

    \item \textbf{\texttt{Handle}}: Invocato per ogni messaggio in ingresso. Riceve il messaggio in input in sola lettura e un puntatore a quello di output pre-allocato, che l'handler deve popolare con il risultato della trasformazione. Questo design evita allocazioni per ogni messaggio e consente all'handler di controllare completamente l'output.

    \item \textbf{\texttt{Close}}: Invocato una sola volta al momento della chiusura dello \textit{stage}, utile per rilasciare risorse acquisite in \texttt{Init}.

    \item \textbf{\texttt{SetTelemetry}}: Utilizzato dalla \textit{pipeline} per fornire all'handler accesso al sottosistema di telemetria, permettendo all'utente di aggiungere log, metriche e \textit{span} di tracciamento personalizzati.
\end{itemize}

Lo \textit{stage} è implementato con tre parametri di tipo generici, consentendo composizioni arbitrarie: ad esempio, uno \textit{stage} potrebbe accettare messaggi \texttt{CannelloniMessage} e produrre messaggi \texttt{CANMessage}, oppure ricevere \texttt{CSVMessage} e produrre un tipo completamente personalizzato.

Quando lo \textit{stage} è eseguito in modalità \textit{worker pool}, ogni \textit{worker} riceve una copia della istanza \texttt{CustomHandler} fornita. Questo significa che lo stato interno dell'handler non è condiviso tra \textit{worker} (a meno che non sia esplicitamente sincronizzato tramite meccanismi concorrenti quali \texttt{sync.Mutex} o canali). Questo design garantisce \textit{thread-safety} per costruzione, ma obbliga l'utente a gestire manualmente lo stato condiviso se necessario.

La libreria fornisce \texttt{CustomHandlerBase}, una classe base che implementa già \texttt{Init}, \texttt{Close} e \texttt{SetTelemetry} con comportamenti di \textit{default} (\textit{no-op} per i primi due, semplice assegnamento per il terzo). Gli utenti possono ereditare da questa classe e implementare solamente il metodo \texttt{Handle}, semplificando il \textit{boilerplate}.

\begin{lstlisting}[language=Go, caption={Esempio implementazione interfaccia CustomHandler}]
type MyHandler struct{
	processor.CustomHandlerBase
}

func (h *MyHandler) Handle(ctx context.Context, msgIn msgInType, msgOut msgOutType) error {
	
    // My custom logic ...
    
    return nil
}
\end{lstlisting}

La configurazione dello \textit{Custom stage} include un campo \texttt{Name} che identifica univocamente lo \textit{stage} nella telemetria e nella tracciatura, consentendo di distinguere tra molteplici \textit{stage} personalizzati nella stessa \textit{pipeline}. Lo \textit{stage} supporta inoltre sia \textit{Single Worker Mode} che \textit{Worker Pool Mode}, fornendo all'utente il controllo sul livello di parallelismo.

In sintesi, il \textit{Custom stage} è lo strumento ideale per scenari in cui la logica di elaborazione è specifica dell'applicazione, dinamica, o semplicemente troppo niche per giustificare l'implementazione di uno \textit{stage} predefinito. Esempi includono: \textit{machine learning inference}, trasformazioni semantiche complesse, aggregazioni statistiche, deduplicazione, arricchimento dati da fonti esterne, o qualunque operazione personalizzata di dominio.

\subsection{Filter}

Lo \textit{Filter stage} è un \textit{Processor stage} generico che applica un predicato di filtraggio ai messaggi in transito nella \textit{pipeline}, decidendo per ciascuno se debba proseguire verso gli \textit{stage} successivi oppure essere scartato. Opera quindi come un filtro basato su una funzione booleana definita dall'utente, che incapsula la logica di accettazione/rifiuto a livello applicativo (ad esempio: tenere solo messaggi con un certo campo valorizzato, scartare quelli di errore, limitare il flusso a un sottoinsieme di segnali rilevanti).

Dal punto di vista dell'esecuzione, il \textit{Filter stage} invoca la funzione di filtro sul messaggio in ingresso: se il predicato restituisce \texttt{false}, il messaggio viene marcato come \textit{dropped} tramite il metodo dedicato nel messaggio e non sarà processato dagli \textit{stage} successivi; se restituisce \texttt{true}, il messaggio viene semplicemente propagato in uscita senza modifiche. In questo modo lo \textit{stage} è completamente trasparente rispetto al contenuto e si limita ad agire come ``valvola'' di selezione.

Grazie all'uso di \textit{generics} sul tipo \texttt{T}, il \textit{Filter stage} può essere inserito in qualunque punto della \textit{pipeline}, indipendentemente dal tipo di messaggio in transito. Tipici esempi includono l'applicazione di filtri su messaggi \textit{CAN} decodificati, su righe \textit{CSV} strutturate, o su eventi provenienti da \textit{Ingress stage} differenti, senza necessità di adattatori intermedi.

\subsection{Tee}

Lo \textit{Tee stage} è un \textit{Processor stage} \textit{utility} specializzato nel duplicare i messaggi verso molteplici output \textit{connector}, consentendo il \textit{branching} della \textit{pipeline}. Il suo nome richiama l'omonimo comando \textit{Unix} \texttt{tee}, che copia lo standard input sia verso lo standard output che verso un file. Analogamente, questo \textit{stage} prende un messaggio in ingresso e lo distribuisce parallelamente verso $N$ output \textit{connector}, permettendo a molteplici sottopipeline di elaborare indipendentemente lo stesso dato.

Un aspetto cruciale del \textit{Tee stage} è che non esegue una copia profonda dei dati. Al contrario, la libreria \textbf{goccia} implementa un sistema di \textit{reference counting} sui messaggi: quando il metodo \texttt{Clone} del messaggio è invocato, viene creato un nuovo envelope che riferisce lo stesso payload sottostante, incrementando un contatore di riferimenti. Solo quando il contatore scende a zero (ossia quando tutti i cloni sono distrutti), il payload effettivo è deallocato. Questo design consente al \textit{Tee stage} di distribuire messaggi a costo praticamente costante, indipendentemente da quanto grandi siano i dati sottostanti. Di fatto vengono copiati solamente i metadati del messaggio, operazione resa possibile dal fatto che gli \textit{stage} a valle utilizzano il messaggio in input come un'istanza \textit{read-only}.

% <codice Clone>

Il \textit{Tee stage} legge un messaggio dal \textit{connector} di input e lo clona per ogni output \textit{connector} configurato. Non vi è alcuna elaborazione o filtraggio: ogni clone è identico all'originale in termini di payload e metadati. Se uno degli output \textit{connector} non riesce a ricevere il clone (ad esempio, è pieno o chiuso), solo quel ramo è interessato; gli altri continuano a ricevere i loro cloni. L'errore è loggato ma non interrompe la distribuzione verso gli altri output.

A differenza dei \textit{Processor stage} visti in precedenza, il \textit{Tee stage} non può utilizzare un \textit{worker pool}. È invece implementato come \textit{stage} \textit{standalone} che gestisce direttamente il ciclo di lettura e distribuzione. La configurazione è minimale: l'utente specifica semplicemente il \textit{connector} di input e un elenco di output \textit{connector} tramite un parametro \textit{variadic}. Al momento dell'inizializzazione, lo \textit{stage} valida che almeno un output \textit{connector} sia stato fornito.

Il \textit{Tee stage} è indispensabile in scenari dove i dati devono essere elaborati in parallelo da molteplici sottosistemi: ad esempio, un singolo flusso di messaggi \textit{CAN} potrebbe essere distribuito simultaneamente a uno \textit{stage} di storage, a uno di analisi \textit{real-time} e a uno di \textit{machine learning inference}. Il \textit{Tee stage} garantisce che ogni sottopipeline riceva una copia coerente del messaggio originale, senza contaminazione tra rami e senza duplicazione effettiva dei dati.

\subsection{Reorder Buffer}

Lo \textit{Reorder Buffer} (ROB) \textit{stage} è specializzato nel riordinamento di messaggi che arrivano fuori ordine, garantendo che vengano elaborati e inoltrati in ordine sequenziale basato su un numero di sequenza contenuto nei messaggi stessi. Questo \textit{stage} può essere eseguito solamente in \textit{Single Worker Mode}, poiché il riordinamento richiede uno stato coerente che non può essere facilmente sincronizzato tra \textit{worker} paralleli.

Il cuore del ROB è un'architettura \textit{dual-buffer}: il \texttt{primary buffer} e l'\texttt{auxiliary buffer}. Il \texttt{primary buffer} è più piccolo ed è ottimizzato per il caso comune dove i messaggi arrivano in ordine o con ritardi brevi. L'\texttt{auxiliary buffer} è più grande e accoglie messaggi che arrivano molto fuori ordine. I due buffer operano in tandem: quando l'\texttt{auxiliary buffer} raggiunge un certo livello di pienezza (configurabile tramite \texttt{FlushTreshold}, \textit{default} 30\%), il \texttt{primary buffer} è interamente liberato nel connettore di output e il contenuto dell'\texttt{auxiliary buffer} è trasferito nel \texttt{primary buffer}.

\begin{figure}[ht]
  \centering
  \includegraphics[width=1\textwidth]{images/library_architecture/rob.png}
  \caption{Schema dell'architettura dual-buffer dello stage ROB}
  \label{fig:rob}
\end{figure}

Quando un messaggio arriva allo \textit{stage}, viene passato al ROB tramite il metodo \texttt{Enqueue}. Il ROB classifica il risultato dell'operazione di enqueue in una di quattro categorie:

\begin{itemize}
    \item \textit{In-Order}: il numero di sequenza corrisponde al prossimo numero atteso. Il messaggio è immediatamente inoltrato al connettore di output senza buffering. Questo rappresenta il caso ideale (\textit{fast path}).
    \item \textit{Primary}: il numero di sequenza è fuori ordine ma rientra nella finestra del \texttt{primary buffer}. Il messaggio è inserito in posizione corretta all'interno del buffer. In questo caso, il messaggio non rientra nel caso ideale, ma l'overhead rimane comunque contenuto.
    \item \textit{Auxiliary}: il numero di sequenza è oltre la finestra del \texttt{primary buffer} ma rientra nella finestra dell'\texttt{auxiliary buffer}. Il messaggio è accodato nell'\texttt{auxiliary buffer}, ricadendo nella peggior casistica in termini di velocità.
    \item \textit{Error}: il numero di sequenza è invalido (duplicato, troppo grande, fuori della finestra complessiva). Il messaggio è scartato e un errore è registrato nelle metriche opportune.
\end{itemize}

\begin{lstlisting}[language=Go, caption={Logica di Enqueue del ROB stage (internal/rob/rob.go, metodo Enqueue)}]
func (rob *ROB[T]) Enqueue(item T) (EnqueueStatus, error) {
	seqNum := item.GetSequenceNumber()

	// Check the sequence number validity
	if !rob.primaryBuf.isValidSize(seqNum) {
		return EnqueueStatusErr, ErrSeqNumTooBig
	}

	// Initialize the ROB if Enqueue is called for the first time ...

	// Try to enqueue the item in the primary buffer
	status, err := rob.enqueuePrimary(item)
	if err == nil {
		return status, nil
	}

	if errors.Is(err, ErrSeqNumDuplicated) {
		return EnqueueStatusErr, err
	}

	// Enqueue the item in the auxiliary buffer
	err = rob.enqueueAuxiliary(item)
	return EnqueueStatusAuxiliary, err
}
\end{lstlisting}

Entrambi i buffer mantengono una finestra di numeri di sequenza validi. La finestra del \texttt{primary buffer} copre il primo range (fino a \texttt{PrimaryBufferSize}), mentre la finestra dell'\texttt{auxiliary buffer} copre da \texttt{PrimaryBufferSize} alla dimensione del buffer ausiliario, \texttt{AuxiliaryBufferSize}. Quando uno dei buffer esaurisce i numeri di sequenza disponibili (ad esempio, tutti i messaggi nella finestra sono stati processati), la finestra si sposta in avanti incrementando il numero di sequenza atteso successivo. Questo è implementato tramite il metodo \texttt{shiftLeft} che compatta i dati nel buffer e azzera gli ultimi slot.

Per gestire efficientemente lo spazio nel buffer e identificare slot vuoti vs occupati, il ROB utilizza una \textit{bitmap}: un array di byte dove ogni bit rappresenta uno slot nel buffer. Operazioni come \texttt{set} e \texttt{isSet} consentono di tracciare quali posizioni contengono messaggi e quali sono vuote con complessità \(O(1)\). Il metodo \texttt{getConsecutive} scansiona la bitmap a partire dall'indice 0 e conta quanti bit consecutivi sono impostati, fornendo il numero di messaggi pronti per il dequeue sequenziale.

Un aspetto importante del ROB è il \textit{timeSmoother}, che implementa un \textit{Double Exponential Smoother} (noto anche come \textit{Holt's Linear Exponential Smoothing}, descritto nella sezione \ref{sec:double-exponential-smoothing}) per regolare e levigare i timestamp associati ai messaggi. A differenza di tecniche di smoothing tradizionali che utilizzano un singolo parametro di smorzamento, il \textit{Double Exponential Smoother} mantiene due componenti di stato: il \textit{level} (stima del valore attuale) e il \textit{trend} (stima della velocità di cambiamento).

Lo smoothing può essere abilitato o disabilitato tramite un apposito parametro di configurazione. Quando abilitato, il smoother applica due fattori di smorzamento: \texttt{EstimatorAlpha} (fattore di smoothing dei dati, \textit{default} 0.8) e \texttt{EstimatorBeta} (fattore di smoothing del trend, \textit{default} 0.5). Il \textit{level} è aggiornato combinando il timestamp osservato con una previsione basata sul \textit{level} e il \textit{trend} precedenti, scalati dalla distanza sequenziale tra il messaggio corrente e quello precedente. Il \textit{trend} è aggiornato combinando il cambiamento nel \textit{level} con il \textit{trend} precedente, permettendo al smoother di adattarsi a variazioni dinamiche nella frequenza di arrivo dei messaggi. Un vincolo di monotonia garantisce che il timestamp aggiustato non retrocede mai nel tempo, preservando la coerenza temporale del flusso di output indipendentemente dall'ordine di arrivo dei messaggi.

\begin{lstlisting}[language=Go, caption={Calcolo del timestamp corretto nel ROB stage (internal/rob/time\_smoother.go, metodo adjust)}]
func (ts *timeSmoother[T]) adjust(item T) {
	// Extract the current receive time
	recvTime := item.GetReceiveTime()
	currValue := float64(recvTime.UnixNano())

	// Extract the current sequence number
	// and the distance to the previous one
	seqNum := item.GetSequenceNumber()
	ts.prevSeqNum = seqNum
	distance := getSeqNumDistance(seqNum, ts.prevSeqNum, ts.maxSeqNum)

	// Force the distance to be at least 1
	if distance == 0 {
		distance = 1
	}

	// Estimate the timestamp value and convert to a timestamp
	estimatedValue := ts.estimator.estimate(currValue, distance)
	currTimestamp := time.Unix(0, int64(estimatedValue))

	// Enforce monotonicity
	if currTimestamp.Before(ts.prevTimestamp) {
		currTimestamp = ts.prevTimestamp
	} else {
		ts.prevTimestamp = currTimestamp
	}

	item.SetTimestamp(currTimestamp)
}
\end{lstlisting}

\begin{lstlisting}[language=Go, caption={Implememtazione Double Exponential Smoothing (internal/rob/time\_smoother.go, metodo estimate)}]
func (dee *doubleExponentialEstimator) estimate(value float64, n uint64) float64 {
	// ...

	// Get the forecasted value based on the previous level and trend
	prevForecasted := dee.prevLevel + dee.prevTrend*float64(n)

	// Calculate the current level and trend to be used
	// by the next item
	currLevel := dee.alpha*value + (1-dee.alpha)*(prevForecasted)
	currTrend := dee.beta*(currLevel-dee.prevLevel) + (1-dee.beta)*dee.prevTrend

	dee.prevLevel = currLevel
	dee.prevTrend = currTrend

	dee.estimateCount++
	return prevForecasted
}
\end{lstlisting}

Il ROB implementa un meccanismo di reset basato su timeout. Durante l'esecuzione, lo \textit{stage} legge i messaggi in ingresso con un timeout pari a \texttt{ResetTimeout} (\textit{default} 100 ms) e se nessun messaggio arriva entro questo periodo, il buffer viene resettato e tutti i messaggi presenti al suo interno vengono inoltrati allo \textit{stage} successivo. Questo protegge da situazioni di deadlock dove un messaggio critico per il riordinamento non arriva mai, causando un accumulo indefinito di messaggi nei buffer. Il reset è inoltre eseguito quando il connettore di input si chiude o quando il contesto di esecuzione è cancellato (\texttt{context.Done}).

Lo \textit{stage} traccia metriche granulari per ogni categoria di enqueue, come il numero di messaggi ricevuti in ordine, quelli fuori ordine salvati nel \texttt{primary}/\texttt{auxiliary buffer}, infine il numero di errori e i reset.

Il ROB \textit{stage} è essenziale in scenari dove il flusso di messaggi in ingresso subisce riordinamento dovuto a latenze variabili o buffering di rete. Esempi includono: raccolta di pacchetti trasmessi con protocolli lossy come \textit{UDP}, i quali potrebbero essere ricevuti non in ordine, o telemetria \textit{real-time} da sistemi remoti dove la variabilità di latenza è significativa.

In sintesi, il ROB \textit{stage} è uno strumento sofisticato che trasforma un flusso potenzialmente disordinato in un flusso ordinato e temporalmente coerente, combinando tecniche di buffering adattivo, bitmap compatte per efficienza spaziale, \textit{Double Exponential Smoothing} statistico, e timeout intelligenti per garantire output di alta qualità anche in condizioni avverse di rete, latenza, e sincronizzazione temporale.

