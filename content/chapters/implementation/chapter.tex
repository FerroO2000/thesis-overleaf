\chapter{Implementazione del sistema di telemetria automobilistica di SquadraCorse}
\label{sec:implementation}

Il sistema di telemetria in tempo reale per il prototipo di Formula SAE di SquadraCorse Polito è stato progettato secondo un’architettura a più livelli che separa chiaramente le responsabilità tra acquisizione, elaborazione, persistenza, osservabilità e visualizzazione dei dati.

L’autovettura, collegata a Internet tramite un gateway 5G, espone due linee CAN (Controller Area Network); i messaggi trasmessi sui due bus vengono incapsulati secondo il protocollo Cannelloni, poi in datagrammi UDP, e inviati tramite una VPN Tailscale verso un server centrale su cui è eseguito un insieme di servizi orchestrati tramite Docker.

Dal punto di vista logico, l’architettura è suddivisa in cinque layer principali:

\begin{description}
  \item[\textbf{Livello di acquisizione dati (\textit{Ingestion Layer})}]%
  Responsabile della ricezione sicura dei datagrammi UDP provenienti dall’auto e del loro instradamento verso i servizi interni.

  \item[\textbf{Livello di elaborazione (\textit{Processing Layer})}]%
  Implementato tramite la libreria \textbf{Goccia}, si occupa di decodificare i frame Cannelloni, interpretare i messaggi CAN e trasformarli in segnali ad alto livello.

  \item[\textbf{Livello di persistenza (\textit{Storage Layer})}]%
  Gestisce la memorizzazione dei dati di telemetria in un database time-series ad alte prestazioni.

  \item[\textbf{Livello di osservabilità (\textit{Observability Layer})}]%
  Raccoglie, struttura e correla metriche e trace relativi al comportamento dell’intero sistema di telemetria, fornendo una base unificata per il monitoraggio tecnico.

  \item[\textbf{Livello di visualizzazione (\textit{Visualization Layer})}]%
  Espone in modo sicuro agli utenti (tramite dashboard e interfacce web) sia i dati di telemetria persistiti sia i segnali di osservabilità, fungendo da punto di accesso unificato per l’analisi in tempo reale e post.
\end{description}

<disegno architettura generale>

\section{Ingestion Layer}
\label{sec:ingestion-layer}

Il \textit{Livello di acquisizione dati} comprende tutti i componenti responsabili di ricevere i messaggi provenienti dall’auto, tramite la rete 5G e la VPN Tailscale, e di recapitarli ai container di telemetria basati su \textbf{Goccia}.

\subsubsection{Connettività Tailscale}

Il container \texttt{tailscale} stabilisce una rete privata virtuale basata su WireGuard fra il gateway 5G a bordo vettura e il server di telemetria. Una volta stabilita la VPN, i datagrammi UDP generati dal gateway vengono incapsulati nel tunnel Tailscale e consegnati all’istanza \texttt{tailscale} in esecuzione sul server, la quale li rende disponibili agli altri servizi Docker.

Il reale mittente dei pacchetti è una PCB custom sviluppata internamente al team ~\cite{scanner:github} che si occupa dell’effettiva lettura dei due bus CAN ed effettua l’incapsulamento secondo il protocollo Cannelloni. Questa scheda è programmata, lato firmware, per inviare i datagrammi UDP verso un unico indirizzo IP, ma su due porte diverse, una per ciascuna linea CAN.

Per poter trasmettere i dati fino al server in sicurezza, il gateway (basato su OpenWRT) configura innanzitutto un alias sull’interfaccia LAN (\texttt{br-lan}), in modo da \textit{simulare un host} con indirizzo IP corrispondente alla destinazione attesa dalla PCB. L’alias viene creato come interfaccia \texttt{lan\_alias} di tipo statico, associata a \texttt{br-lan} e configurata con l’indirizzo, ad esempio, \texttt{192.168.10.254/32}.

Successivamente, il gateway abilita l’inoltro IPv4 e definisce delle regole di \textit{DNAT} nella tabella di \texttt{PREROUTING} che riscrive la destinazione di tutto il traffico proveniente da \texttt{br-lan} e indirizzato a \texttt{192.168.10.254} verso l’indirizzo IP del container Tailscale del server.

In questo modo, la PCB continua a inviare i pacchetti verso un indirizzo IP locale fisso (l’alias sulla LAN), mentre il gateway, in maniera trasparente, li inoltra attraverso la VPN Tailscale verso il server di telemetria. Il firmware della scheda resta così indipendente dai dettagli di configurazione della VPN e dell’infrastruttura remota.

\subsubsection{Instradamento dei datagram}

Il servizio \texttt{udp-proxy} è il punto di demarcazione tra la VPN e il resto dello stack di telemetria. Esso condivide il medesimo namespace di rete del container Tailscale (tramite \texttt{network\_mode: service:tailscale}) e riceve quindi direttamente i datagrammi UDP provenienti dalla vettura.

La sua funzione è quella di \textit{instradare} i pacchetti UDP verso i corretti consumer a valle, in base alla porta di destinazione. Per ciascuna porta di ascolto configurata viene creata un’istanza del proxy, che apre una socket UDP in ascolto e apre una connessione verso l’endpoint che si occuperà del processamento.

Il cuore del componente è implementato utilizzando il server UDP offerto dalla libreria standard di Go.

\section{Processing Layer}
\label{sec:processing-layer}

Il \textit{Livello di elaborazione} è implementato dal servizio denominato \texttt{sc-telemetry}~\cite{sc-telemetry:github}, eseguito in due istanze distinte per i due bus CAN. Ciascuna istanza realizza, tramite la libreria \textbf{Goccia}, una pipeline di elaborazione a sei stadi che parte dai datagrammi UDP incapsulati in Cannelloni e termina con la persistenza dei segnali CAN in QuestDB.

L’immagine \texttt{scomarferro/sc-telemetry}~\cite{sc-telemetry:dockerhub} incapsula in un unico binario Go l’implementazione necessaria per gestire il flusso di dati proveniente dal veicolo. In fase di avvio, il servizio carica la configurazione leggendo un file YAML (di default \texttt{/app/config/config.yaml}) e applica eventuali override tramite variabili d’ambiente. Il percorso del file di configurazione può essere modificato impostando la variabile \texttt{CONFIG\_PATH}, mentre ogni campo del file YAML può essere sovrascritto da un corrispondente \textit{environment variable}. Questa strategia consente di mantenere una configurazione di base versionata nel repository e, al contempo, di adattare rapidamente i parametri di runtime al contesto specifico senza ricompilare il binario.

La pipeline implementata da \texttt{sc-telemetry} è composta da sei stadi principali:
\begin{itemize}
  \item \textbf{UDP Ingress}: riceve i datagrammi UDP provenienti dalla scheda SCanner tramite il proxy UDP e li inserisce nella pipeline.
  \item \textbf{Cannelloni Decoder Processor}: prende in ingresso i payload dei pacchetti UDP e decodifica il payload secondo la specifica del protocollo Cannelloni.
  \item \textbf{ROB Processor (Re-Order Buffer)}: ordina i messaggi dello stage precedente in base al numero di sequenza, compensando il fatto che i datagrammi UDP possono arrivare fuori ordine o con \textit{jitter} significativo. Oltre al riordino, lo stadio corregge i timestamp associati ai messaggi applicando tecniche di smoothing temporale descritte nei capitoli precedenti, in modo da stimare con maggiore precisione l’istante effettivo di generazione del messaggio a bordo vettura.
  \item \textbf{CAN Processor}: prende in ingresso i messaggi Cannelloni riordinati, estrae i frame CAN grezzi e li decodifica secondo le specifiche contenute nel file DBC~\cite{dbc:format}. Questo tipo di file rappresenta lo standard del settore automobilistico per definire la struttura dei messaggi trasmessi nelle linee CAN. Il file DBC viene letto, in fase di inizializzazione, dalla libreria \texttt{acmelib}~\cite{acmelib:github}.
  \item \textbf{CAN Message Handler (Custom Processor)}: è l’unico stadio della pipeline implementato specificamente per questa applicazione, e ha il compito di trasformare i messaggi CAN decodificati in messaggi pronti per essere inseriti in QuestDB.
  \item \textbf{QuestDB Egress}: si occupa di inserire i messaggi nel database time-series.
\end{itemize}

\begin{figure}[ht]
  \centering
  \includegraphics[width=1\textwidth]{images/implementation/pipeline.png}
  \caption{Stage di Goccia usati in sc-telemetry}
  \label{fig:sc-telemetry-pipeline}
\end{figure}
\section{Storage Layer}
\label{sec:storage-layer}

Il \textit{Livello di persistenza} ha il compito di memorizzare in modo durevole e strutturato i dati di telemetria prodotti dalle pipeline, rendendoli disponibili sia per la visualizzazione quasi real--time sia per analisi differite.

\subsubsection{QuestDB}

QuestDB è stato scelto come backend di persistenza poiché è un database time-series open source progettato specificamente per carichi ad alto throughput su dati indicizzati temporalmente. A differenza dei database relazionali tradizionali, adotta uno storage column-oriented e un modello dati nativamente temporale: il timestamp è un tipo primitivo e può fungere da chiave di partizionamento, consentendo l’organizzazione fisica delle tabelle per intervalli temporali (ad esempio per giorno). Questo approccio, combinato con tecniche di compressione e partizionamento, permette di gestire l’ingestion di milioni di record al secondo e query analitiche su grandi volumi di dati mantenendo latenze contenute.

L’interfaccia di accesso si basa su SQL esteso con primitive specifiche per le serie temporali, come \texttt{SAMPLE BY} per il downsampling, \texttt{LATEST ON} per il recupero dello stato più recente, che risultano particolarmente adatti alle analisi di dati telemetrici. Inoltre, il supporto ai protocolli standard come PGwire (compatibile PostgreSQL) e alle API HTTP/REST facilita l’integrazione con strumenti esterni e con la piattaforma Grafana. Nel contesto di questa architettura, QuestDB agisce come single source of truth per i dati di telemetria, mentre i dati di osservabilità relativi al funzionamento del sistema (metriche e trace) sono affidati ai backend specializzati Prometheus e Tempo.

\subsubsection{Data Model}

Nel modello dati adottato per QuestDB, i segnali CAN decodificati vengono suddivisi in più tabelle in base al \textit{tipo di valore} associato al segnale. In particolare, esistono quattro tabelle principali: \texttt{flag\_signals} per i segnali booleani, \texttt{int\_signals} per i segnali interi, \texttt{float\_signals} per i segnali in virgola mobile ed \texttt{enum\_signals} per i segnali enumerativi. Questa scelta consente di mantenere omogenee le colonne di ciascuna tabella e di semplificare sia l’ingestione sia le query analitiche successive.

Tutte le tabelle condividono un insieme di colonne comuni: un campo simbolico \texttt{name}, che identifica il segnale (ad esempio \texttt{motor\_rpm} o \texttt{coolant\_temp}), il campo intero \texttt{can\_id}, che memorizza l’ID del frame CAN da cui il segnale è stato estratto, e il campo intero \texttt{raw\_value}, che conserva il valore grezzo così come presente nel payload del frame. A queste colonne comuni si affiancano poi colonne specifiche per ciascun tipo: nella tabella \texttt{flag\_signals} è presente la colonna booleana \texttt{flag\_value}; in \texttt{int\_signals} la colonna intera \texttt{integer\_value}; in \texttt{float\_signals} la colonna a virgola mobile \texttt{float\_value}. Per i segnali enumerativi, memorizzati in \texttt{enum\_signals}, oltre a \texttt{can\_id} e \texttt{raw\_value} viene utilizzato anche un simbolo \texttt{enum\_value}, che rappresenta la label dell’enumerazione corrispondente al valore numerico sottostante.

\begin{lstlisting}[language=SQL, caption={Definizione tabella dei segnali in QuestDB}]
CREATE TABLE float_signals (
  timestamp TIMESTAMP,
  name SYMBOL,
  can_id LONG,
  raw_value LONG,
  /*
  integer_value LONG for int_signals table
  flag_value BOOLEAN for flag_signals table
  enum_value SYMBOL for enum_signals table
  */
  float_value DOUBLE
) TIMESTAMP(timestamp) PARTITION BY DAY;
\end{lstlisting}

In questo modo, l’organizzazione dei dati in QuestDB risulta sia normalizzata rispetto al tipo del segnale, sia sufficientemente ricca da consentire analisi flessibili: le colonne \texttt{name} ed eventuali simboli aggiuntivi (\texttt{enum\_value}) permettono di filtrare e aggregare per segnali logici di alto livello, mentre \texttt{can\_id} e \texttt{raw\_value} preservano il legame con la rappresentazione originaria sul bus CAN.

\subsubsection{Throughput, partizionamento e retention}

Dal punto di vista prestazionale, QuestDB è dimensionato per gestire senza difficoltà il volume di dati generato dai due bus CAN del prototipo SquadraCorse. La documentazione ufficiale e benchmark indipendenti riportano capacità di ingest nell’ordine dei milioni di record al secondo su hardware moderno, valori ben superiori al carico previsto in questa applicazione.\cite{questdb:architecture}\cite{questdb:benchmark}\cite{griddb:benchmark}

La strategia di partizionamento temporale adottata (tipicamente per giorno) consente di implementare in modo semplice politiche di retention basate su orizzonti temporali, eliminando intere partizioni in un’unica operazione, e al tempo stesso riduce la quantità di dati che devono essere scansionati dalle query, che nella pratica si concentrano spesso su singole sessioni di test o giornate di prove. Ne risulta una migliore località dei dati su disco, con effetti positivi sulle latenze di interrogazione.

Infine, la presenza di QuestDB come layer esclusivamente \textit{di storage} per i dati di telemetria -- separato dallo storage di metriche (Prometheus) e trace (Tempo) -- rende l’architettura più modulare: è possibile, ad esempio, estendere in futuro il sistema con ulteriori database (per analisi offline o machine learning) senza impattare sul percorso critico di acquisizione ed elaborazione dei segnali provenienti dalla vettura.

\section{Observability Layer}
\label{sec:observability-layer}

Il \textit{Livello di osservabilità} raccoglie e struttura i segnali relativi al comportamento del sistema (metriche e trace), indipendentemente da come verranno successivamente visualizzati. In questa architettura è realizzato da tre componenti principali: \texttt{OpenTelemetry Collector}~\cite{otel:collector}, \texttt{Prometheus}~\cite{prometheus:official}, \texttt{Tempo}~\cite{grafana:tempo} e \texttt{Grafana}.

\subsubsection{OpenTelemetry Collector}

Questo è il componente general-purpose per la ricezione, il processamento e l’esportazione di segnali di osservabilità provenienti da più sorgenti.

Nel sistema di telemetria di SquadraCorse, il Collector riceve tramite protocollo OTLP/gRPC i trace e le metriche prodotti dalle istanze di \texttt{sc-telemetry}, strumentate tramite \textbf{Goccia}; su questi dati applica quindi una catena di \textit{processor} (ad esempio un \texttt{batch} per raggruppare gli eventi, un \texttt{memory\_limiter} per controllare il consumo di memoria ed eventualmente processor di \textit{sampling} per ridurre il volume dei trace) e, infine, esporta le metriche verso Prometheus ed i trace verso Grafana Tempo.

\subsubsection{Prometheus}

Prometheus viene utilizzato come database di serie temporali per le metriche del sistema di telemetria. Raccoglie principalmente le metriche esposte dall’OpenTelemetry Collector, che fornisce una vista aggregata sul comportamento dei vari servizi, con la possibilità di integrare in modo incrementale ulteriori sorgenti (ad esempio metriche esposte direttamente da altri componenti).

Adotta un modello dati dimensionale, in cui ogni serie temporale è identificata da un nome di metrica e da un insieme di label; ciò consente di distinguere, ad esempio, le metriche per servizio, istanza o stage della pipeline e di formulare query che filtrano o aggregano i dati (throughput per stage, latenza p99, numero di errori, ecc.) con grande flessibilità.

\subsubsection{Grafana Tempo}

Per il \textit{distributed tracing} viene impiegato \texttt{Grafana Tempo}, un backend progettato per memorizzare e interrogare trace di applicazioni distribuite a partire da segnali OpenTelemetry, Jaeger o Zipkin.

Nel sistema proposto, Tempo riceve i trace esportati dall’OpenTelemetry Collector. Ogni messaggio di telemetria elaborato da \texttt{sc-telemetry} genera una root span associata alla ricezione del datagramma UDP, mentre ciascuno degli stage della pipeline \textbf{Goccia} (UDP ingress, decoder Cannelloni, ROB, decoder CAN, handler, egress QuestDB) contribuisce con uno span figlio. Ciò permette di ricostruire il percorso completo di elaborazione di un messaggio end-to-end, misurare la latenza introdotta da ciascun stage e individuare colli di bottiglia o anomalie.

Grafana Tempo è particolarmente adatto a questo scenario perché evita l’uso di database di ricerca generici (come Elasticsearch) e utilizza storage ottimizzati per trace compressi, riducendo i costi operativi e mantenendo al contempo capacità di interrogazione avanzate tramite TraceQL. Inoltre, esso è nativamente integrato in Grafana, semplificando la configurazione di quest’ultimo.

\section{Visualization Layer}
\label{sec:visualization-layer}

Il \textit{Livello di visualizzazione} ha il compito di rendere disponibili, in modo sicuro e fruibile, i dati di telemetria e i segnali di osservabilità agli utenti umani (ingegneri di pista, sviluppatori, data analyst). In questa architettura è realizzato principalmente da \texttt{Grafana}~\cite{grafana:official}, come interfaccia di consultazione unificata, e da \texttt{Caddy}~\cite{caddy:docs}, come reverse proxy e terminazione TLS verso l’esterno.

\subsubsection{Grafana}

Grafana costituisce il frontend di riferimento per la visualizzazione dei dati prodotti dagli altri layer del sistema. Tramite opportuni data source, interroga QuestDB per accedere ai segnali CAN decodificati e memorizzati nel Livello di persistenza, Prometheus per le metriche di sistema (throughput, latenze, errori, stato dei servizi) e Grafana Tempo per l’esplorazione dei trace distribuiti. Le dashboard realizzate a supporto dell’attività di SquadraCorse combinano questi sorgenti informativi per offrire, in un’unica interfaccia, una vista completa sia dello stato della vettura sia del comportamento del sistema di telemetria.

La definizione delle dashboard non avviene manualmente tramite l’interfaccia grafica, ma segue un approccio \textit{infrastructure as code}: le configurazioni sono descritte in Jsonnet~\cite{jsonnet:official} utilizzando la libreria Grafonnet~\cite{grafonnet:github}, che fornisce primitive ad alto livello per la costruzione di pannelli, row e datasource. I file Jsonnet vengono quindi compilati in manifest JSON standard di Grafana e inclusi nel processo di provisioning, così da poter versionare non solo il codice applicativo, ma anche la struttura e il contenuto dei dashboard. Questo approccio semplifica la manutenzione e l’evoluzione delle viste, consente di riutilizzare componenti comuni tra dashboard diversi e garantisce la riproducibilità dell’ambiente di visualizzazione su installazioni differenti.

\begin{figure}[H]
  \centering
  \includegraphics[width=0.7\textwidth]{images/implementation/grafana_pressure.JPG}
  \caption{Dashboard Grafana tratte dal sistema di telemetria di SquadraCorse (pressioni)}
  \label{fig:grafana-pressure}
\end{figure}

\begin{figure}[H]
  \centering
  \includegraphics[width=0.7\textwidth]{images/implementation/grafana_cells.JPG}
  \caption{Dashboard Grafana tratte dal sistema di telemetria di SquadraCorse (tensioni/temperature)}
  \label{fig:grafana-cells}
\end{figure}

\begin{figure}[H]
  \centering
  \includegraphics[width=0.7\textwidth]{images/implementation/grafana_fsm.JPG}
  \caption{Dashboard Grafana tratte dal sistema di telemetria di SquadraCorse (FSM)}
  \label{fig:grafana-fsm}
\end{figure}

\subsubsection{Caddy}

Il container Caddy funge da reverse proxy e da punto di terminazione TLS per i servizi web esposti dal sistema, in primo luogo Grafana. In ascolto sulla porta 443/TCP, Caddy si occupa di gestire automaticamente i certificati HTTPS (ad esempio tramite Let’s Encrypt), riducendo in modo significativo la complessità operativa legata alla sicurezza del trasporto, e di instradare le richieste in ingresso verso l’istanza di Grafana esposta sulla rete interna.

Questa separazione tra la responsabilità di visualizzazione, demandata a Grafana, e quella di pubblicazione sicura verso l’esterno, affidata a Caddy, permette di mantenere tutti i componenti interni --- inclusi Prometheus, Tempo e QuestDB --- confinati in reti Docker private, esponendo all’esterno un unico endpoint HTTPS gestito e controllato. In tal modo, il Livello di visualizzazione rimane facilmente accessibile ai membri del team senza introdurre superfici di attacco aggiuntive o accoppiamenti stretti con la topologia interna del sistema di telemetria.
