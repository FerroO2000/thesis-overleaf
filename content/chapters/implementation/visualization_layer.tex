\section{Visualization Layer}
\label{sec:visualization-layer}

Il \textit{Livello di visualizzazione} ha il compito di rendere disponibili, in modo sicuro e fruibile, i dati di telemetria e i segnali di osservabilità agli utenti umani (ingegneri di pista, sviluppatori, data analyst). In questa architettura è realizzato principalmente da \texttt{Grafana}~\cite{grafana:official}, come interfaccia di consultazione unificata, e da \texttt{Caddy}~\cite{caddy:docs}, come reverse proxy e terminazione TLS verso l’esterno.

\subsubsection{Grafana}

Grafana costituisce il frontend di riferimento per la visualizzazione dei dati prodotti dagli altri layer del sistema. Tramite opportuni data source, interroga QuestDB per accedere ai segnali CAN decodificati e memorizzati nel Livello di persistenza, Prometheus per le metriche di sistema (throughput, latenze, errori, stato dei servizi) e Grafana Tempo per l’esplorazione dei trace distribuiti. Le dashboard realizzate a supporto dell’attività di SquadraCorse combinano questi sorgenti informativi per offrire, in un’unica interfaccia, una vista completa sia dello stato della vettura sia del comportamento del sistema di telemetria.

La definizione delle dashboard non avviene manualmente tramite l’interfaccia grafica, ma segue un approccio \textit{infrastructure as code}: le configurazioni sono descritte in Jsonnet~\cite{jsonnet:official} utilizzando la libreria Grafonnet~\cite{grafonnet:github}, che fornisce primitive ad alto livello per la costruzione di pannelli, row e datasource. I file Jsonnet vengono quindi compilati in manifest JSON standard di Grafana e inclusi nel processo di provisioning, così da poter versionare non solo il codice applicativo, ma anche la struttura e il contenuto dei dashboard. Questo approccio semplifica la manutenzione e l’evoluzione delle viste, consente di riutilizzare componenti comuni tra dashboard diversi e garantisce la riproducibilità dell’ambiente di visualizzazione su installazioni differenti.

\begin{figure}[H]
  \centering
  \includegraphics[width=0.7\textwidth]{images/implementation/grafana_pressure.JPG}
  \caption{Dashboard Grafana tratte dal sistema di telemetria di SquadraCorse (pressioni)}
  \label{fig:grafana-pressure}
\end{figure}

\begin{figure}[H]
  \centering
  \includegraphics[width=0.7\textwidth]{images/implementation/grafana_cells.JPG}
  \caption{Dashboard Grafana tratte dal sistema di telemetria di SquadraCorse (tensioni/temperature)}
  \label{fig:grafana-cells}
\end{figure}

\begin{figure}[H]
  \centering
  \includegraphics[width=0.7\textwidth]{images/implementation/grafana_fsm.JPG}
  \caption{Dashboard Grafana tratte dal sistema di telemetria di SquadraCorse (FSM)}
  \label{fig:grafana-fsm}
\end{figure}

\subsubsection{Caddy}

Il container Caddy funge da reverse proxy e da punto di terminazione TLS per i servizi web esposti dal sistema, in primo luogo Grafana. In ascolto sulla porta 443/TCP, Caddy si occupa di gestire automaticamente i certificati HTTPS (ad esempio tramite Let’s Encrypt), riducendo in modo significativo la complessità operativa legata alla sicurezza del trasporto, e di instradare le richieste in ingresso verso l’istanza di Grafana esposta sulla rete interna.

Questa separazione tra la responsabilità di visualizzazione, demandata a Grafana, e quella di pubblicazione sicura verso l’esterno, affidata a Caddy, permette di mantenere tutti i componenti interni --- inclusi Prometheus, Tempo e QuestDB --- confinati in reti Docker private, esponendo all’esterno un unico endpoint HTTPS gestito e controllato. In tal modo, il Livello di visualizzazione rimane facilmente accessibile ai membri del team senza introdurre superfici di attacco aggiuntive o accoppiamenti stretti con la topologia interna del sistema di telemetria.
